\documentclass[12pt]{article}

\usepackage[margin=1in, left=0.6in, right=0.6in]{geometry}
\usepackage{fancyhdr}	% header
\usepackage{hyperref} % links

\usepackage{amsmath,amsthm,amssymb}	%math stuff
\usepackage{graphicx} \graphicspath{ {./images/} }
\usepackage{setspace} % increase line spacing
\usepackage{tabularx} % long tables
\usepackage{enumitem} % labelling itmes
\usepackage{color, soul}
\usepackage{lmodern} % bolding \texttt{}
\usepackage[T1]{fontenc} % for {} in \texttt{}
\usepackage{listings}
\usepackage[table]{xcolor}
\usepackage[edges]{forest}
% \usepackage{array}
% \usepackage{booktabs}
% \usepackage{siunitx}
% \usepackage{alltt}

\definecolor{dkgreen}{rgb}{0,0.6,0}
\definecolor{gray}{rgb}{0.5,0.5,0.5}
\definecolor{mauve}{rgb}{0.58,0,0.82}
\definecolor{backcolour}{rgb}{0.95,0.95,0.92}

\setlength{\parindent}{0pt}

\pagestyle{fancy}
\fancyhead[LO,L]{CSCC37 A1}
\fancyhead[CO,C]{Stephen Guo}
\fancyhead[RO,R]{1006313231}
\fancyfoot[LO,L]{}
\fancyfoot[CO,C]{\thepage}
\fancyfoot[RO,R]{}

\newcommand{\N}{\mathbb{N}}
\newcommand{\R}{\mathbb{R}}
\newcommand{\Rplus}{\mathbb{R}^{+}}
\newcommand{\bigbracket}[1]{\big(#1\big)}
\newcommand{\Bigbracket}[1]{\Big(#1\Big)}
\newcommand{\floorSurround}[1]{\left\lfloor#1\right\rfloor}
\newcommand{\ceilingSurround}[1]{\left\lceil#1\right\rceil}
\newcommand{\code}[1]{{\ttfamily \fontseries{b}\selectfont #1}}
\definecolor{codegray}{gray}{0.9}
\def \calO {\mathcal{O}}
\newcommand{\bigO}[1]{\ensuremath{\calO(#1)}}
\renewcommand{\qed}{\hfill$\blacksquare$}
\newenvironment{proofindent}{\vspace*{2mm}\hfill\begin{minipage}{\dimexpr\textwidth-10mm}}{\end{minipage}}

\everymath{\displaystyle}

\begin{document}
%----------------------------------------------------------------------------------
%                              Table of Contents
%----------------------------------------------------------------------------------
\begin{center}
	\hypertarget{toc}{\LARGE \noindent \underline{\textbf{Table of Contents}}}\\
\end{center}

\noindent \hyperlink{1}{\textbf{Question 1:}}
\vspace{1mm}
\hrule
\vspace{1mm} \leavevmode \\

\noindent \hyperlink{2}{\textbf{Question 2:}}
\vspace{1mm}
\hrule
\vspace{1mm} \leavevmode \\

\noindent \hyperlink{3}{\textbf{Question 3:}}
\vspace{1mm}
\hrule
\vspace{1mm} \leavevmode \\

\noindent \hyperlink{4}{\textbf{Question 4:}}
\vspace{1mm}
\hrule
\vspace{1mm} \leavevmode \\

\noindent \hyperlink{5}{\textbf{Question 5:}}
\vspace{1mm}
\hrule
\vspace{1mm} \leavevmode \\
\newpage

%{\setstretch{1.5}$\begin{array}{r@{}>{\displaystyle}l}  \end{array}$}
% {\setstretch{1.5}$\begin{array}{r@{}>{\displaystyle}l}
% 	&{} \\
% 	&{} \\
% 	&{} \\
% 	&{} \\
% 	&{} \\
% \end{array}$}
%----------------------------------------------------------------------------------
%                                   Questions
%----------------------------------------------------------------------------------
\setstretch{1.2}
%----------------------------------------------------------------------------------
% !                                    1
%----------------------------------------------------------------------------------
\noindent \hyperlink{toc}{\hypertarget{1}{\LARGE \underline{\textbf{Question 1.}}}}\\\\
The most-significant digit of the sum is a $\#$ in the $b^3$ column. However, we are only adding two 2-digit numbers, which cannot be more than $2\times b^3$\\
This means that $\# = 1$.\\

So we have\\
% \begin{tabular}{lS}
% 	    & 1 $*$          \\
% 	$+$ & 1 $*$          \\
% 	\hline %or \bottomrule if using the `booktabs` package
% 	    & 1 $\lozenge$ 1 \\
% \end{tabular}\\
\[
	\begin{array}{@{}cccc}
		  &   & 1        & * \\
		+ &   & 1        & * \\
		\hline
		= & 1 & \lozenge & 1
	\end{array}
\]
In the $b^1$ column, two identical digits adding up to one must mean that the sum carries over, so \\
$* + * = b + 1$\\
This also means that $b$ is odd, since $* + *$ must be even.\\

In the  $b^2$ column, we have a carry over from the $b^1$ column. So \\
$1+1+1 = b+\lozenge $\\
Since $LS$ is odd, and $b$ is odd, this means $\lozenge$ is even.\\
Since $1+1+1$ carries over, this means that $b<=3$. And since $b$ is odd, then $b = 3 \Longrightarrow \lozenge = 0$


~\\\\Answer:
\[
	\begin{array}{@{}cccc}
		  &   & 1 & 2 \\
		+ &   & 1 & 2 \\
		\hline
		= & 1 & 0 & 1
	\end{array}
\]
Where $b = 3$

\newpage
%----------------------------------------------------------------------------------
% !                                    2
%----------------------------------------------------------------------------------
\noindent \hyperlink{toc}{\hypertarget{2}{\LARGE \underline{\textbf{Question 2.}}}}\\\\
\begin{minipage}[t]{0.5\textwidth}
	$(0.1)_{10} = ()_2?$
	\begin{center}
		\begin{tabular}{|c|c|c|c|c|}
			\hline \cellcolor{gray!25}Multiplier &
			\cellcolor{gray!25}Base              &
			\cellcolor{gray!25}Product           &
			\cellcolor{gray!25}Integral          &
			\cellcolor{gray!25}Fraction                                                                       \\
			\hline\hline
			\texttt{0.1}                         & \texttt{2} & \texttt{0.2} & \texttt{0} & \hl{\texttt{0.2}} \\\hline
			\texttt{0.2}                         & \texttt{2} & \texttt{0.4} & \texttt{0} & \texttt{0.4}      \\\hline
			\texttt{0.4}                         & \texttt{2} & \texttt{0.8} & \texttt{0} & \texttt{0.8}      \\\hline
			\texttt{0.8}                         & \texttt{2} & \texttt{1.6} & \texttt{1} & \texttt{0.6}      \\\hline
			\texttt{0.6}                         & \texttt{2} & \texttt{1.2} & \texttt{1} & \hl{\texttt{0.2}} \\\hline
		\end{tabular}
	\end{center}
	$(0.1)_{10} = ()_3?$
	\begin{center}
		\begin{tabular}{|c|c|c|c|c|}
			\hline \cellcolor{gray!25}Multiplier &
			\cellcolor{gray!25}Base              &
			\cellcolor{gray!25}Product           &
			\cellcolor{gray!25}Integral          &
			\cellcolor{gray!25}Fraction                                                                       \\
			\hline\hline
			\texttt{0.1}                         & \texttt{3} & \texttt{0.3} & \texttt{0} & \hl{\texttt{0.3}} \\\hline
			\texttt{0.3}                         & \texttt{3} & \texttt{0.9} & \texttt{0} & \texttt{0.9}      \\\hline
			\texttt{0.9}                         & \texttt{3} & \texttt{2.7} & \texttt{2} & \texttt{0.7}      \\\hline
			\texttt{0.7}                         & \texttt{3} & \texttt{2.1} & \texttt{2} & \texttt{0.1}      \\\hline
			\texttt{0.1}                         & \texttt{3} & \texttt{0.3} & \texttt{0} & \hl{\texttt{0.3}} \\\hline
		\end{tabular}
	\end{center}
	$(0.1)_{10} = ()_4?$
	\begin{center}
		\begin{tabular}{|c|c|c|c|c|}
			\hline \cellcolor{gray!25}Multiplier &
			\cellcolor{gray!25}Base              &
			\cellcolor{gray!25}Product           &
			\cellcolor{gray!25}Integral          &
			\cellcolor{gray!25}Fraction                                                                       \\
			\hline\hline
			\texttt{0.1}                         & \texttt{4} & \texttt{0.4} & \texttt{0} & \hl{\texttt{0.4}} \\\hline
			\texttt{0.4}                         & \texttt{4} & \texttt{1.6} & \texttt{1} & \texttt{0.6}      \\\hline
			\texttt{0.6}                         & \texttt{4} & \texttt{2.4} & \texttt{2} & \hl{\texttt{0.4}} \\\hline
		\end{tabular}
	\end{center}
	$(0.1)_{10} = ()_5?$
	\begin{center}
		\begin{tabular}{|c|c|c|c|c|}
			\hline \cellcolor{gray!25}Multiplier &
			\cellcolor{gray!25}Base              &
			\cellcolor{gray!25}Product           &
			\cellcolor{gray!25}Integral          &
			\cellcolor{gray!25}Fraction                                                                       \\
			\hline\hline
			\texttt{0.1}                         & \texttt{5} & \texttt{0.5} & \texttt{0} & \hl{\texttt{0.5}} \\\hline
			\texttt{0.5}                         & \texttt{5} & \texttt{2.5} & \texttt{2} & \hl{\texttt{0.5}} \\\hline
		\end{tabular}
	\end{center}
\end{minipage}
\begin{minipage}[t]{0.5\textwidth}
	$(0.1)_{10} = ()_6?$
	\begin{center}
		\begin{tabular}{|c|c|c|c|c|}
			\hline \cellcolor{gray!25}Multiplier &
			\cellcolor{gray!25}Base              &
			\cellcolor{gray!25}Product           &
			\cellcolor{gray!25}Integral          &
			\cellcolor{gray!25}Fraction                                                                       \\
			\hline\hline
			\texttt{0.1}                         & \texttt{6} & \texttt{0.6} & \texttt{0} & \hl{\texttt{0.6}} \\\hline
			\texttt{0.6}                         & \texttt{6} & \texttt{3.6} & \texttt{3} & \hl{\texttt{0.6}} \\\hline
		\end{tabular}
	\end{center}
	$(0.1)_{10} = ()_7?$
	\begin{center}
		\begin{tabular}{|c|c|c|c|c|}
			\hline \cellcolor{gray!25}Multiplier &
			\cellcolor{gray!25}Base              &
			\cellcolor{gray!25}Product           &
			\cellcolor{gray!25}Integral          &
			\cellcolor{gray!25}Fraction                                                                       \\
			\hline\hline
			\texttt{0.1}                         & \texttt{7} & \texttt{0.7} & \texttt{0} & \hl{\texttt{0.7}} \\\hline
			\texttt{0.7}                         & \texttt{7} & \texttt{4.9} & \texttt{4} & \texttt{0.9}      \\\hline
			\texttt{0.9}                         & \texttt{7} & \texttt{6.3} & \texttt{6} & \texttt{0.3}      \\\hline
			\texttt{0.3}                         & \texttt{7} & \texttt{2.1} & \texttt{2} & \texttt{0.1}      \\\hline
			\texttt{0.1}                         & \texttt{7} & \texttt{0.7} & \texttt{0} & \hl{\texttt{0.7}} \\\hline
		\end{tabular}
	\end{center}
	$(0.1)_{10} = ()_8?$
	\begin{center}
		\begin{tabular}{|c|c|c|c|c|}
			\hline \cellcolor{gray!25}Multiplier &
			\cellcolor{gray!25}Base              &
			\cellcolor{gray!25}Product           &
			\cellcolor{gray!25}Integral          &
			\cellcolor{gray!25}Fraction                                                                       \\
			\hline\hline
			\texttt{0.1}                         & \texttt{8} & \texttt{0.8} & \texttt{0} & \hl{\texttt{0.8}} \\\hline
			\texttt{0.8}                         & \texttt{8} & \texttt{6.4} & \texttt{6} & \texttt{0.4}      \\\hline
			\texttt{0.4}                         & \texttt{8} & \texttt{3.2} & \texttt{3} & \texttt{0.2}      \\\hline
			\texttt{0.2}                         & \texttt{8} & \texttt{1.6} & \texttt{1} & \texttt{0.6}      \\\hline
			\texttt{0.6}                         & \texttt{8} & \texttt{4.8} & \texttt{4} & \hl{\texttt{0.8}} \\\hline
		\end{tabular}
	\end{center}
	$(0.1)_{10} = ()_9?$
	\begin{center}
		\begin{tabular}{|c|c|c|c|c|}
			\hline \cellcolor{gray!25}Multiplier &
			\cellcolor{gray!25}Base              &
			\cellcolor{gray!25}Product           &
			\cellcolor{gray!25}Integral          &
			\cellcolor{gray!25}Fraction                                                                       \\
			\hline\hline
			\texttt{0.1}                         & \texttt{9} & \texttt{0.9} & \texttt{0} & \hl{\texttt{0.9}} \\\hline
			\texttt{0.9}                         & \texttt{9} & \texttt{8.1} & \texttt{8} & \texttt{0.1}      \\\hline
			\texttt{0.1}                         & \texttt{9} & \texttt{0.9} & \texttt{0} & \hl{\texttt{0.9}} \\\hline
		\end{tabular}
	\end{center}
\end{minipage}\\

In every base from 2 to 9, the decimal expansion loops. $\therefore (0.1)_{10}$ cannot be represented exactly with a finite mantissa.
\newpage
%----------------------------------------------------------------------------------
% !                                    3
%----------------------------------------------------------------------------------
\noindent \hyperlink{toc}{\hypertarget{3}{\LARGE \underline{\textbf{Question 3.}}}}\\\\
\[\delta = \frac{x - fl(x)}{x} \]
To find an upper bound, we need to see what's the maximum value of $x - fl(x)$\\
We want to prove that $\delta$ is bounded above by\\
{\setstretch{1.5}$\begin{array}{r@{}>{\displaystyle}l}
		\epsilon & {}= \left\{
		\begin{aligned}
			\ b^{1-t}            & \hspace*{3mm}\text{ chopping } \\
			\ \frac{1}{2}b^{1-t} & \hspace*{3mm}\text{ rounding } \\
		\end{aligned}
		\right.                \\
		         & {}=         \\
		         & {}=         \\
	\end{array}$}\\
Let $x = \pm\ d_k \times b^k + d_{k-1} \times b^{k-1} + d_{k-2} \times b^{k-2} + \ldots$\\
Then $fl(x) = \pm\ \overbrace{d_k \times b^k + d_{k-1} \times b^{k-1} + d_{k-2} \times b^{k-2} + \ldots + d_{k-t-1} \times b^{k-t-1}}^{t \text{ values}}$\\

$x - fl(x) = d_{k-t-2} \times b^{k-t-2} + d_{k-t-3} \times b^{k-t-3} + \ldots $\\
Which is bounded above by $$
	\left\{
	\begin{aligned}
		\ 1 \times b^{k-t-1}           & \hspace*{3mm}\text{ chopping } & \hspace*{3mm} \text{[max value is 1 decimal place above]}            \\
		\ \frac{1}{2} \times b^{k-t-1} & \hspace*{3mm}\text{ rounding } & \hspace*{3mm} \text{[since adding $\frac{1}{2}$ reduces reduces RRO]} \\
	\end{aligned}
	\right.
$$

if $t$ is the mantissa length, then the amount of significant digits $b^{k-t-1}$ has is $b^1-t$. So $$
\epsilon = \left\{
	\begin{aligned}
		\ 1 \times b^{k-t-1}           & \hspace*{3mm}\text{ chopping } \\
		\ \frac{1}{2} \times b^{k-t-1} & \hspace*{3mm}\text{ rounding } \\
	\end{aligned}
\right.
$$

\newpage
%----------------------------------------------------------------------------------
% !                                    4
%----------------------------------------------------------------------------------
\noindent \hyperlink{toc}{\hypertarget{4}{\LARGE \underline{\textbf{Question 4.}}}}\\\\
\newpage
%----------------------------------------------------------------------------------
% !                                    5
%----------------------------------------------------------------------------------
\noindent \hyperlink{toc}{\hypertarget{5}{\LARGE \underline{\textbf{Question 5.}}}}\\\\
\end{document}