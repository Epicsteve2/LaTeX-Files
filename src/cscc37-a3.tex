\documentclass[12pt]{article}

\usepackage[margin=1in, left=0.6in, right=0.6in]{geometry}
\usepackage{fancyhdr}	% header
\usepackage{hyperref} % links

\usepackage{amsmath,amsthm,amssymb}	%math stuff
\usepackage{graphicx} \graphicspath{ {./images/} }
\usepackage{setspace} % increase line spacing
\usepackage{tabularx} % long tables
\usepackage{enumitem} % labelling itmes
\usepackage{color, soul}
\usepackage{lmodern} % bolding \texttt{}
\usepackage[T1]{fontenc} % for {} in \texttt{}
\usepackage{listings}
\usepackage[table]{xcolor}
\usepackage[edges]{forest}
\usepackage{xfrac} % slanted fractions
% \usepackage{array}
% \usepackage{booktabs}
% \usepackage{siunitx}
% \usepackage{alltt}

\definecolor{dkgreen}{rgb}{0,0.6,0}
\definecolor{gray}{rgb}{0.5,0.5,0.5}
\definecolor{mauve}{rgb}{0.58,0,0.82}
\definecolor{backcolour}{rgb}{0.95,0.95,0.92}

\setlength{\parindent}{0pt}

\pagestyle{fancy}
\fancyhead[LO,L]{CSCC37 A3}
\fancyhead[CO,C]{Stephen Guo}
\fancyhead[RO,R]{1006313231}
\fancyfoot[LO,L]{}
\fancyfoot[CO,C]{\thepage}
\fancyfoot[RO,R]{}

\newcommand{\N}{\mathbb{N}}
\newcommand{\R}{\mathbb{R}}
\newcommand{\Rplus}{\mathbb{R}^{+}}
\newcommand{\bigbracket}[1]{\big(#1\big)}
\newcommand{\Bigbracket}[1]{\Big(#1\Big)}
\newcommand{\floorSurround}[1]{\left\lfloor#1\right\rfloor}
\newcommand{\ceilingSurround}[1]{\left\lceil#1\right\rceil}
\newcommand{\code}[1]{{\ttfamily \fontseries{b}\selectfont #1}}
\definecolor{codegray}{gray}{0.9}
\def \calO {\mathcal{O}}
\newcommand{\bigO}[1]{\ensuremath{\calO(#1)}}
\renewcommand{\qed}{\hfill$\blacksquare$}
\newenvironment{proofindent}{\vspace*{2mm}\hfill\begin{minipage}{\dimexpr\textwidth-10mm}}{\end{minipage}}

\everymath{\displaystyle}

\begin{document}
%----------------------------------------------------------------------------------
%                              Table of Contents
%----------------------------------------------------------------------------------
\begin{center}
    \hypertarget{toc}{\LARGE \noindent \underline{\textbf{Table of Contents}}}\\
\end{center}

\noindent {\textbf{Question 1:}}
\vspace{1mm}
\hrule
\vspace{1mm}
\noindent
\hyperlink{1.1}{(a)}\\
\hyperlink{1.2}{(b)}\\
\hyperlink{1.3}{(c)}\\
\hyperlink{1.4}{(d)}\\

\noindent \textbf{Question 2:}
\vspace{1mm}
\hrule
\vspace{1mm}
\hyperlink{2.1}{(a)}\\
\hyperlink{2.2}{(b)}\\

\noindent \textbf{Question 3:}
\vspace{1mm}
\hrule
\vspace{1mm}
\noindent
\hyperlink{3.1}{(a)}\\
\hyperlink{3.2}{(b)}\\
\hyperlink{3.3}{(c)}\\

\noindent \hyperlink{4}{\textbf{Question 4:}}
\vspace{1mm}
\hrule
\vspace{1mm} \leavevmode \\

\noindent {\textbf{Question 5:}}
\vspace{1mm}
\hrule
\vspace{1mm}
\noindent
\hyperlink{5.1}{(a)}\\
\hyperlink{5.2}{(b)}\\
\hyperlink{5.3}{(c)}\\
\hyperlink{5.4}{(d)}\\

\newpage

%{\setstretch{1.5}$\begin{array}{r@{}>{\displaystyle}l}  \end{array}$}
% {\setstretch{1.5}$\begin{array}{r@{}>{\displaystyle}l}
% 	&{} \\
% 	&{} \\
% 	&{} \\
% 	&{} \\
% 	&{} \\
% \end{array}$}
%----------------------------------------------------------------------------------
%                                   Questions
%----------------------------------------------------------------------------------
\setstretch{1.2}
%----------------------------------------------------------------------------------
% !                                    1
%----------------------------------------------------------------------------------
\noindent \hyperlink{toc}{\LARGE \underline{\textbf{Question 1.}}}\\\\
\noindent ~\\\hyperlink{toc}{\hypertarget{1.1}{(a)}}\\
If there is an interval $[a,\ b]$ such that
\begin{enumerate}
    \item $g(x) \in [a,\ b] \qquad \forall x\in [a,\ b]$
    \item $|g'(x)| \leq L < 1 \qquad \forall x\in [a,\ b]$
\end{enumerate}
Then $g(x)$ has a unique fixed point in $[a,\ b]$

\noindent ~\\\hyperlink{toc}{\hypertarget{1.2}{(b)}}\\
Suppose 1. and 2. \\
Start with any $x_0\in [a,\ b]$ and iterate\\
$x_{k+1} = g(x_k) \qquad k = 1,\ 2,\ \cdots$\\

Then $x_k \in [a,\ b]$ by 1.\\

Moreover,
{
        \setstretch{1.5}
        $$
            \begin{array}{r@{}>{\displaystyle}ll}
                x_{k+1} - x_{k}
                 & {}= g(x_{k}) - x_{k-1}          & \\
                 & {}= g'(\eta_k)(x_{k+1} - x_{k}) &
            \end{array}
        $$
        \begin{center}
            for some $\eta_k \in [x_{k-1},\ x_k] \subset [a,\ b] \qquad$ (condition 2)
        \end{center}
    }
So
$$
    \begin{array}{r@{}>{\displaystyle}ll}
        x_{k+1} - x_{k}
         & {}\leq L |x_{k} - x_{k-1}|     & \\
         & {}\leq L^2 |x_{k-1} - x_{k-2}| & \\
         & {}\leq \hspace*{15mm}\vdots                  & \\
         & {}\leq L^{k-1} |x_{2} - x_{1}| & \\
         & {}\leq L^k |x_{1} - x_{0}|     & \\
    \end{array}
$$
$L_k$ approaches 0 as $k$ appraoches $\infty$. So $|x_{1} - x_{0}|$ approaches 0 as well\\
$\therefore x_k$ converges to some point $\widetilde{x}\in [a,\ b]$\\

\noindent ~\\\hyperlink{toc}{\hypertarget{1.3}{(c)}}\\
\noindent ~\\\hyperlink{toc}{\hypertarget{1.4}{(d)}}\\
\newpage
%----------------------------------------------------------------------------------
% !                                    2
%----------------------------------------------------------------------------------
\noindent \hyperlink{toc}{\LARGE \underline{\textbf{Question 2.}}}\\\\
\noindent ~\\\hyperlink{toc}{\hypertarget{2.1}{(a)}}\\
\noindent ~\\\hyperlink{toc}{\hypertarget{2.2}{(b)}}\\
\newpage
%----------------------------------------------------------------------------------
% !                                    3
%----------------------------------------------------------------------------------
\noindent \hyperlink{toc}{{\LARGE \underline{\textbf{Question 3.}}}}\\\\
\noindent ~\\\hyperlink{toc}{\hypertarget{3.1}{(a)}}\\
\noindent ~\\\hyperlink{toc}{\hypertarget{3.2}{(b)}}\\
\noindent ~\\\hyperlink{toc}{\hypertarget{3.3}{(c)}}\\
\newpage
%----------------------------------------------------------------------------------
% !                                    4
%----------------------------------------------------------------------------------
\noindent \hyperlink{toc}{\hypertarget{4}{\LARGE \underline{\textbf{Question 4.}}}}\\\\
\newpage
%----------------------------------------------------------------------------------
% !                                    5
%----------------------------------------------------------------------------------
\noindent \hyperlink{toc}{{\LARGE \underline{\textbf{Question 5.}}}}\\\\
\noindent ~\\\hyperlink{toc}{\hypertarget{5.1}{(a)}}\\
\noindent ~\\\hyperlink{toc}{\hypertarget{5.2}{(b)}}\\
\noindent ~\\\hyperlink{toc}{\hypertarget{5.3}{(c)}}\\
\noindent ~\\\hyperlink{toc}{\hypertarget{5.4}{(d)}}\\
\newpage
\end{document}