\documentclass[12pt]{article}

\usepackage[margin=1in, left=0.6in, right=0.6in]{geometry}
\usepackage{fancyhdr} % header
\usepackage{hyperref} % links

\usepackage{amsmath,amsthm,amssymb}	%math stuff
\usepackage{graphicx} \graphicspath{ {./images/} }
\usepackage{setspace} % increase line spacing
\usepackage{tabularx} % long tables
\usepackage{enumitem} % labelling itmes
\usepackage{color, soul}
\usepackage{lmodern} % bolding \texttt{}
\usepackage[T1]{fontenc} % for {} in \texttt{}
\usepackage{listings}
\usepackage[table]{xcolor}
\usepackage[edges]{forest}
\usepackage{xfrac} % slanted fractions
\usepackage{changepage}   % for the adjustwidth environment
\usepackage{multirow} % merging 2 rows
\usepackage{relsize} % Scaling the font

% \usepackage{array}
% \usepackage{booktabs}
% \usepackage{siunitx}
% \usepackage{alltt}

\definecolor{dkgreen}{rgb}{0,0.6,0}
\definecolor{gray}{rgb}{0.5,0.5,0.5}
\definecolor{mauve}{rgb}{0.58,0,0.82}
\definecolor{backcolour}{rgb}{0.95,0.95,0.92}

\setlength{\parindent}{0pt}

\pagestyle{fancy}
\fancyhead[LO,L]{CSCC37 A4}
\fancyhead[CO,C]{Stephen Guo}
\fancyhead[RO,R]{1006313231}
\fancyfoot[LO,L]{}
\fancyfoot[CO,C]{\thepage}
\fancyfoot[RO,R]{}

\newcommand{\N}{\mathbb{N}}
\newcommand{\R}{\mathbb{R}}
\newcommand{\Rplus}{\mathbb{R}^{+}}
\newcommand{\bigbracket}[1]{\big(#1\big)}
\newcommand{\Bigbracket}[1]{\Big(#1\Big)}
\newcommand{\floorSurround}[1]{\left\lfloor#1\right\rfloor}
\newcommand{\ceilingSurround}[1]{\left\lceil#1\right\rceil}
\newcommand{\code}[1]{{\ttfamily \fontseries{b}\selectfont #1}}
\definecolor{codegray}{gray}{0.9}
\def \calO {\mathcal{O}}
\newcommand{\bigO}[1]{\ensuremath{\calO(#1)}}
\renewcommand{\qed}{\hfill$\blacksquare$}
\newenvironment{proofindent}{\vspace*{2mm}\hfill\begin{minipage}{\dimexpr\textwidth-10mm}}{\end{minipage}}

\everymath{\displaystyle}

\begin{document}
%----------------------------------------------------------------------------------
%                              Table of Contents
%----------------------------------------------------------------------------------
\begin{center}
    \hypertarget{toc}{\LARGE \underline{\textbf{Table of Contents}}}\\
\end{center}

{\textbf{Question 1:}}
\vspace{1mm}
\hrule
\vspace{1mm}
\hyperlink{1.1}{(a)}\\
\hyperlink{1.2}{(b)}\\
\hyperlink{1.3}{(c)}\\
\hyperlink{1.4}{(d)}\\
\hyperlink{1.5}{(e)}\\
\hyperlink{1.6}{(f)}\\

\textbf{Question 2:}
\vspace{1mm}
\hrule
\vspace{1mm}
\hyperlink{2.1}{(a)}\\
\hyperlink{2.2}{(b)}\\

\textbf{Question 3:}
\vspace{1mm}
\hrule
\vspace{1mm}
\hyperlink{3.1}{(a)}\\
\hyperlink{3.2}{(b)}\\

\hyperlink{4}{\textbf{Question 4:}}
\vspace{1mm}
\hrule
\vspace{1mm} \leavevmode \\

\newpage

% {\setstretch{1.5}$\begin{array}{r@{}>{\displaystyle}l}  \end{array}$}

% {
%     \setstretch{2.5}
%     $
%         \begin{array}{r@{}>{\displaystyle}l}
%          & {}= \\
%          & {}= \\
%          & {}= \\
%         \end{array}
%     $
% }

%----------------------------------------------------------------------------------
%                                   Questions
%----------------------------------------------------------------------------------
\setstretch{1.2}
%----------------------------------------------------------------------------------
% !                                     1
%----------------------------------------------------------------------------------
{\LARGE\underline{\textbf{Question 1.}}}\\
~\\\hyperlink{toc}{\hypertarget{1.1}{(a)}}\\
{
\setstretch{1.5}
$
    \begin{array}{r@{}>{\displaystyle}l}
        p(-1) & {}= {4} \\
        p(0)  & {}={6}  \\
        p(1)  & {}={12} \\
    \end{array}
$
}
$\Longleftrightarrow$
$
    \begin{bmatrix}
        1 & -1 & 1 \\
        1 & 0  & 0 \\
        1 & 1  & 1 \\
    \end{bmatrix}
    \begin{bmatrix}
        a_0 \\a_1\\a_2
    \end{bmatrix}
    =
    \begin{bmatrix}
        4 \\6\\12
    \end{bmatrix}
$\\

Eliminate 1$^{\text{st}}$ column:\\
\begin{minipage}[t]{0.5\textwidth}
    $$
        P_1 = I =
        \begin{bmatrix}
            1 & 0 & 0 \\
            0 & 1 & 0 \\
            0 & 0 & 1 \\
        \end{bmatrix}
    $$
\end{minipage}
\begin{minipage}[t]{0.5\textwidth}
    $$
        P_1 A= A =
        \begin{bmatrix}
            1 & -1 & 1 \\
            1 & 0  & 0 \\
            1 & 1  & 1 \\
        \end{bmatrix}
    $$
\end{minipage}\\

\begin{minipage}[t]{0.5\textwidth}
    $$
        L_1 =
        \begin{bmatrix}
            1  & 0 & 0 \\
            -1 & 1 & 0 \\
            -1 & 0 & 1 \\
        \end{bmatrix}
    $$
\end{minipage}
\begin{minipage}[t]{0.5\textwidth}
    $$
        L_1P_1A =
        \begin{bmatrix}
            1 & -1 & 1  \\
            0 & 1  & -1 \\
            0 & 2  & 0  \\
        \end{bmatrix}
    $$
\end{minipage}\\

Eliminate 2$^{\text{nd}}$ column:\\
\begin{minipage}[t]{0.5\textwidth}
    $$
        P_2 =
        \begin{bmatrix}
            1 & 0 & 0 \\
            0 & 0 & 1 \\
            0 & 1 & 0 \\
        \end{bmatrix}
    $$
\end{minipage}
\begin{minipage}[t]{0.5\textwidth}
    $$
        P_2 L_1P_1A =
        \begin{bmatrix}
            1 & -1 & 1  \\
            0 & 2  & 0  \\
            0 & 1  & -1 \\
        \end{bmatrix}
    $$
\end{minipage}\\
\begin{minipage}[t]{0.5\textwidth}
    $$
        L_2 =
        \begin{bmatrix}
            1 & 0             & 0 \\
            0 & 1             & 0 \\
            0 & -\sfrac{1}{2} & 1 \\
        \end{bmatrix}
    $$
\end{minipage}
\begin{minipage}[t]{0.5\textwidth}
    $$
        L_2 P_2 L_1P_1A =
        \begin{bmatrix}
            1 & -1 & 1  \\
            0 & 2  & 0  \\
            0 & 0  & -1 \\
        \end{bmatrix}
    $$
\end{minipage}\\\\

{
\setstretch{1.5}
$
    \begin{array}{rrr@{}l}
        L_2 P_2 L_1P_1A = U
         & \Longleftrightarrow & L_2 P_2 L_1P_2P_2P_1A      & {}= U                               \\
         & \Longleftrightarrow & L_2 (P_2 L_1P_2)P_2P_1A    & {}= U                               \\
         & \Longleftrightarrow & L_2 \widetilde{L_1}P_2P_1A & {}= U                               \\
         & \Longleftrightarrow & P_2P_1A                    & {}= \widetilde{L_1}^{-1} L_2^{-1} U \\
         & \Longleftrightarrow & PA                         & {}= L U                             \\
    \end{array}
$
}\newpage

\begin{minipage}[t]{0.5\textwidth}
    $
        \widetilde{L_1} = P_2 L_1P_2 =
        \begin{bmatrix}
            1  & 0 & 0 \\
            -1 & 1 & 0 \\
            -1 & 0 & 1 \\
        \end{bmatrix}
    $
\end{minipage}
\begin{minipage}[t]{0.5\textwidth}
    $
        L = \widetilde{L_1}^{-1} L_2^{-1} =
        \begin{bmatrix}
            1 & 0            & 0 \\
            1 & 1            & 0 \\
            1 & \sfrac{1}{2} & 1 \\
        \end{bmatrix}
    $
\end{minipage}\\





$
    P = P_2P_1 =
    \begin{bmatrix}
        1 & 0 & 0 \\
        0 & 0 & 1 \\
        0 & 1 & 0 \\
    \end{bmatrix}
$\\

$$
    \begin{bmatrix}
        1 & 0 & 0 \\
        0 & 0 & 1 \\
        0 & 1 & 0 \\
    \end{bmatrix}
    \begin{bmatrix}
        1 & -1 & 1 \\
        1 & 0  & 0 \\
        1 & 1  & 1 \\
    \end{bmatrix}
    =
    \begin{bmatrix}
        1 & 0            & 0 \\
        1 & 1            & 0 \\
        1 & \sfrac{1}{2} & 1 \\
    \end{bmatrix}
    \begin{bmatrix}
        1 & -1 & 1  \\
        0 & 2  & 0  \\
        0 & 0  & -1 \\
    \end{bmatrix}
$$

{
        \setstretch{1.5}
        $$
            \begin{array}{r@{}>{\displaystyle}l}
                A\vec{x} = \vec{b} & {}\Longleftrightarrow PA\vec{x} = P\vec{b}   \\
                                   & {}\Longleftrightarrow LU\vec{x} = P\vec{b}   \\
                                   & {}\Longleftrightarrow L(U\vec{x}) = P\vec{b} \\
                                   & {}\Longleftrightarrow L\vec{d} = P\vec{b}    \\
            \end{array}
        $$

        $$
            P\vec{b} =
            \begin{bmatrix}
                4 \\12\\6
            \end{bmatrix}
        $$
    }\newpage

Forward solve $L\vec{d} = \vec{b}$ for $\vec{d}$:\\
$
    \begin{bmatrix}
        1 & 0            & 0 \\
        1 & 1            & 0 \\
        1 & \sfrac{1}{2} & 1 \\
    \end{bmatrix}
    \begin{bmatrix}
        d_1 \\d_2\\d_3
    \end{bmatrix}
    =
    \begin{bmatrix}
        4 \\12\\6
    \end{bmatrix}
$\\
$d_1 = 4$\\
$d_1 + d_2 = 12$\\
$d_1 + \frac{1}{2}d_2 + d_3 = 6$\\

$d_1 = 4$\\
$d_2 = 8$\\
$d_3 = -2$\\

Backward solve $U\vec{x} = \vec{d}$ for $\vec{x}$\\
$
    \begin{bmatrix}
        1 & -1 & 1  \\
        0 & 2  & 0  \\
        0 & 0  & -1 \\
    \end{bmatrix}
    \begin{bmatrix}
        x_1 \\x_2\\x_3
    \end{bmatrix}
    =
    \begin{bmatrix}
        4 \\8\\-2
    \end{bmatrix}
$\\

$x_3 = 2$\\
$x_2 = 4$\\
$x_1 = 6$\\

$\therefore p(x) = 6 + 4x + 2x^2$
\newpage\hyperlink{toc}{\hypertarget{1.2}{(b)}}\\
$$
    \displaystyle l_i(x) = \prod_{\substack{j=0\\j\not=i}}^{n}\frac{x-x_j}{x_i-x_j}
    \qquad i =0,\ 1,\ 2
$$
\begin{center}
    {
    \setstretch{1.5}
    $
        \begin{array}{r@{}>{\displaystyle}l}
            x_{0} & {}= -1 \\
            x_{1} & {}= 0  \\
            x_{2} & {}= 1  \\
        \end{array}
    $
    }
    {
    \setstretch{1.5}
    $
        \begin{array}{r@{}>{\displaystyle}l}
            y_{0} & {}= 4  \\
            y_{1} & {}= 6  \\
            y_{2} & {}= 12 \\
        \end{array}
    $
    }
\end{center}~\\\\

\begin{minipage}[t]{0.3333333333333333333333333\textwidth}
    \begin{center}
        $i=0$
    \end{center}
    $j = 1,\ 2$\\
    {
    \setstretch{2.5}
    $
        \begin{array}{r@{}>{\displaystyle}l}
            l_{0}(x) & {}= \left(\frac{x-0}{-1-0}\right)\left(\frac{x-1}{-1-1}\right) \\
                     & {}= \left(-x\right)\left(\frac{x-1}{-2}\right)                 \\
                     & {}= \frac{1}{2} x(x-1)                                         \\
        \end{array}
    $
    }
\end{minipage}
\begin{minipage}[t]{0.3333333333333333333333333\textwidth}
    \begin{center}
        $i=1$
    \end{center}
    $j = 0,\ 2$\\
    {
    \setstretch{2.5}
    $
        \begin{array}{r@{}>{\displaystyle}l}
            l_{1}(x) & {}= \left(\frac{x-(-1)}{0-(-1)}\right)\left(\frac{x-1}{0-1}\right) \\
                     & {}= (x+1)(1-x)                                                     \\
        \end{array}
    $
    }
\end{minipage}
\begin{minipage}[t]{0.3333333333333333333333333\textwidth}
    \begin{center}
        $i=2$
    \end{center}
    $j = 0,\ 1$\\
    {
    \setstretch{2.5}
    $
        \begin{array}{r@{}>{\displaystyle}l}
            l_{2}(x) & {}= \left(\frac{x-(-1)}{1-(-1)}\right)\left(\frac{x-0}{1-0}\right) \\
                     & {}= \left(\frac{x+1}{2}\right)(x)                                  \\
                     & {}=\frac{1}{2}x(x+1)                                               \\
        \end{array}
    $
    }
\end{minipage}\\\\

{
\setstretch{2.5}
$$
    \begin{array}{r@{}>{\displaystyle}l}
        p(x) & {}= \sum^{n}_{i=0}l_i(x)y_i                                       \\
             & {}= l_0(x)y_0 + l_1(x)y_1 + l_2(x)y_2                             \\
             & {}= \frac{1}{2} x(x-1)(4) + (x+1)(1-x)(6) + \frac{1}{2}x(x+1)(12) \\
             & {}= 2 x^2-2x + (-6x^2) + 6 + 6x^2 + 6x                            \\
             & {}= 2 x^2+4x + 6                                                  \\
    \end{array}
$$
}

\newpage\hyperlink{toc}{\hypertarget{1.3}{(c)}}\\
\begin{center}
    \begin{tabular}{|c|c|c|c|}
        \hline
        $x_i$                 & $y[x_i]$                  & $y[x_{i+1},\ x_i]$                                    & $y[x_{i+2},\ x_{i+1},\ x_i]$                          \\\hline\hline
        $\multirow{2}{*}{-1}$ & $\multirow{2}{*}{\hl{4}}$ & \multicolumn{1}{c}{}                                  &                                                       \\\cline{3-3}
                              &                           & \multirow{2}{*}{$\frac{6-4}{1-(-1)} = \text{\hl{2}}$} &                                                       \\\cline{0-1}\cline{4-4}
        $\multirow{2}{*}{0}$  & $\multirow{2}{*}{6}$      &                                                       & \multirow{2}{*}{$\frac{6-2}{1-(-1)} = \text{\hl{2}}$} \\\cline{3-3}
                              &                           & \multirow{2}{*}{$\frac{12-6}{1-0} = 6$}               &                                                       \\\cline{0-1}\cline{4-4}
        $\multirow{2}{*}{1}$  & $\multirow{2}{*}{12}$     &                                                       &                                                       \\\cline{3-3}
                              &                           & \multicolumn{1}{c}{}                                  &                                                       \\\hline
    \end{tabular}
\end{center}
{
\setstretch{1.5}
$$
    \begin{array}{r@{}>{\displaystyle}l}
        p(x) & {}= y[x_0] + (x-x_0)y[x_1,\ x_0] + (x-x_0)(x-x_1)y[x_2,\ x_1,\ x_0] \\
             & {}= 4 + \big(x-(-1)\big)(2) + \big(x-(-1)\big)(x-0)(2)              \\
             & {}= 4 + 2x + 2 + 2x^2 + 2x                                          \\
             & {}= 6 + 4x + 2x^2                                                   \\
    \end{array}
$$
}

~\\\hyperlink{toc}{\hypertarget{1.4}{(d)}}\\
As all the equations are simplified from (a), (b), and (c), we can see that all polynomials are identical.\\

\newpage\hyperlink{toc}{\hypertarget{1.5}{(e)}}\\
We can use the method in (c) since all we have to do is to add to the table\\
\begin{center}
    \begin{tabular}{|c|c|c|c|c|}
        \hline
        $x_i$                 & $y[x_i]$                  & $y[x_{i+1},\ x_i]$                                    & $y[x_{i+2},\ x_{i+1},\ x_i]$                          & $y[x_{i+3},\ x_{i+2},\ x_{i+1},\ x_i]$                  \\\hline\hline
        $\multirow{2}{*}{-1}$ & $\multirow{2}{*}{\hl{4}}$ & \multicolumn{1}{c}{}                                  & \multicolumn{1}{c}{}                                  &                                                         \\\cline{3-3}
                              &                           & \multirow{2}{*}{$\frac{6-4}{1-(-1)} = \text{\hl{2}}$} & \multicolumn{1}{c}{}                                  &                                                         \\\cline{0-1}\cline{4-4}
        $\multirow{2}{*}{0}$  & $\multirow{2}{*}{6}$      &                                                       & \multirow{2}{*}{$\frac{6-2}{1-(-1)} = \text{\hl{2}}$} &                                                         \\\cline{3-3}\cline{5-5}
                              &                           & \multirow{2}{*}{$\frac{12-6}{1-0} = 6$}               &                                                       & \multirow{2}{*}{$\frac{-1-2}{2-(-1)} = \text{\hl{-1}}$} \\\cline{0-1}\cline{4-4}
        $\multirow{2}{*}{1}$  & $\multirow{2}{*}{12}$     &                                                       & \multirow{2}{*}{$\frac{4-6}{2-0} = -1$}               &                                                         \\\cline{3-3}\cline{5-5}
                              &                           & \multirow{2}{*}{$\frac{16-12}{2-1} = 4$}              &                                                       &                                                         \\\cline{0-1}\cline{4-4}
        $\multirow{2}{*}{2}$  & $\multirow{2}{*}{16}$     &                                                       & \multicolumn{1}{c}{}                                  &                                                         \\\cline{3-3}
                              &                           & \multicolumn{1}{c}{}                                  & \multicolumn{1}{c}{}                                  &                                                         \\\hline
    \end{tabular}
\end{center}

$$
    \begin{array}{r@{}>{\displaystyle}l}
        p(x) & {}= y[x_0] + (x-x_0)y[x_1,\ x_0]                                                        \\
             & {} \hspace*{14.25mm} + (x-x_0)(x-x_1)y[x_2,\ x_1,\ x_0]                                 \\
             & {} \hspace*{14.25mm} + (x-x_0)(x-x_1)(x-x_2)y[x_3,\ x_2,\ x_1,\ x_0]                    \\
             & {}= 4 + \big(x-(-1)\big)(2) + \big(x-(-1)\big)(x-0)(2) + \big(x-(-1)\big)(x-0)(x-1)(-1) \\
             & {}= 4 + 2x + 2 + 2x^2 + 2x -x^3 + x                                                     \\
             & {}= 6 + 5x + 2x^2 -x^3                                                                  \\
    \end{array}
$$
\newpage\hyperlink{toc}{\hypertarget{1.6}{(f)}}\\
Equation of line from $(-1,\ 4)$ to $(0,\ 6)$:\\
\begin{minipage}[t]{0.5\textwidth}
    {
        \setstretch{2.5}
        $
            \begin{array}{r@{}>{\displaystyle}l}
                \text{slope}_1
                 & {}= \frac{\text{rise}}{\text{run}} \\
                 & {}= \frac{y_2-y_1}{x_2-x_1}        \\
                 & {}= \frac{6-4}{0-(-1)}             \\
                 & {}= 2                              \\
            \end{array}
        $
    }
\end{minipage}
\begin{minipage}[t]{0.5\textwidth}
    {
        \setstretch{2.5}
        $
            \begin{array}{r@{}>{\displaystyle}l}
                y-y_1=\text{slope}_1(x-x_1)
                 & {}\Longleftrightarrow y-4=2\big(x-(-1)\big) \\
                 & {}\Longleftrightarrow y=2(x+1) + 4          \\
                 & {}\Longleftrightarrow y=2x + 6              \\
            \end{array}
        $
    }
\end{minipage}\\\\

Equation of line from $(0,\ 6)$ to $(1,\ 12)$:\\
\begin{minipage}[t]{0.5\textwidth}
    {
        \setstretch{2.5}
        $
            \begin{array}{r@{}>{\displaystyle}l}
                \text{slope}_1
                 & {}= \frac{\text{rise}}{\text{run}} \\
                 & {}= \frac{y_2-y_1}{x_2-x_1}        \\
                 & {}= \frac{12-6}{1-0}               \\
                 & {}= 6                              \\
            \end{array}
        $
    }
\end{minipage}
\begin{minipage}[t]{0.5\textwidth}
    {
        \setstretch{2.5}
        $
            \begin{array}{r@{}>{\displaystyle}l}
                y-y_1=\text{slope}_1(x-x_1)
                 & {}\Longleftrightarrow y-6=6(x-0) \\
                 & {}\Longleftrightarrow y=6x + 6   \\
            \end{array}
        $
    }
\end{minipage}\\\\

Equation of line from $(1,\ 12)$ to $(2,\ 16)$:\\
\begin{minipage}[t]{0.5\textwidth}
    {
        \setstretch{2.5}
        $
            \begin{array}{r@{}>{\displaystyle}l}
                \text{slope}_1
                 & {}= \frac{\text{rise}}{\text{run}} \\
                 & {}= \frac{y_2-y_1}{x_2-x_1}        \\
                 & {}= \frac{16-12}{2-1}              \\
                 & {}= 4                              \\
            \end{array}
        $
    }
\end{minipage}
\begin{minipage}[t]{0.5\textwidth}
    {
        \setstretch{2.5}
        $
            \begin{array}{r@{}>{\displaystyle}l}
                y-y_1=\text{slope}_1(x-x_1)
                 & {}\Longleftrightarrow y-12=4(x-1) \\
                 & {}\Longleftrightarrow y=4x + 8    \\
            \end{array}
        $
    }
\end{minipage}\\

$\therefore$ the equation of the linear spline is:
$$
    \left\{
    \begin{aligned}
        \ y=2x + 6 & \hspace*{13mm} \text{if } -1 \leq x < 0   \\
        \ y=6x + 6 & \hspace*{13mm} \text{if } 0 \leq x < 1    \\
        \ y=4x + 8 & \hspace*{13mm} \text{if } 1 \leq x \leq 2 \\
    \end{aligned}
    \right.
$$

% $y-y_1=\text{slope}_1(x-x_1)$\\
\newpage
%----------------------------------------------------------------------------------
% !                                     2
%----------------------------------------------------------------------------------
{\LARGE \underline{\textbf{Question 2.}}}\\
~\\\hyperlink{toc}{\hypertarget{2.1}{(a)}}\\
When we solve using the Vandermonde method, we get a polynomial of the form:\\
$$p(x) = a_0 + a_1x + a_2{x}^2 + a_3{x}^3 + \ldots + a_{n-1}{x}^{n-1} + a_{n}{x}^{n}$$\\

We can factor this polynomial into the form:
$$p(x) = a_0 + x\bigg(a_1 + x \Big(a_2 + x\big(a_3 + \ldots + x(a_{n-1} + xa_n) \cdots \big)\Big)\bigg)$$
There are $n-1$ additions, and $n-1$ multiplications, so we have\\
$2n + \mathcal{O}(1)$ flops.

% For addition, we have $n-1$ flops\\

% For mulltiplication, we can expand $p(x)$ to be of the form:
% $$p(x) = a_0 + a_1x + a_2{x}{x} + a_3{x}{x}{x} + \ldots + a_{n-1}\overbrace{{x}\cdots{x}}^{n-1 \text{ times}} + a_{n}\overbrace{{x}\cdots{x}}^{n \text{ times}}$$

% So for the amount of multiplication pairs, we have
% $$0 + 1 + 2 + \ldots + (n-1) + (n)$$

% For total flops, we have:
% {
%     \setstretch{2.5}
%     $$
%         \begin{array}{r@{}>{\displaystyle}l}
%             \text{total flops} & {}= \text{addition flops + multiplication flops} \\
%              & {}= (n-1) + \left(\sum_{n=1}^{n}n\right)\\
%              & {}= (n-1) + \frac{1}{2}(n)(n+1)\\
%              & {}= \frac{n^2}{2} + \frac{3n}{2} + 1\\
%              & {}= \frac{n^2}{2} + \mathcal{O}(n)\\
%         \end{array}
%     $$
% }
\newpage\hyperlink{toc}{\hypertarget{2.2}{(b)}}\\
For divided-difference, we get a polynomial of the form:
$$
    \begin{array}{r@{}>{\displaystyle}l}
        p(x) = y[x_0] & {} + (x-x_0)y[x_1,\ x_0]                                                                  \\
                      & {} + (x-x_0)(x-x_1)y[x_2,\ x_1,\ x_0]                                                     \\
                      & {} + (x-x_0)(x-x_1)(x-x_2)y[x_3,\ x_2,\ x_1,\ x_0]                                        \\
                      & {} \hspace*{2mm}\vdots                                                                    \\
                      & {} + (x-x_0)(x-x_1)\cdots(x-x_{n-3})(x-x_{n-2})y[x_{n-1},\ x_{n-2},\ \cdots ,\ x_1,\ x_0] \\
                      & {} + (x-x_0)(x-x_1)\cdots(x-x_{n-2})(x-x_{n-1})y[x_{n},\ x_{n-1},\ \cdots ,\ x_1,\ x_0]   \\
    \end{array}
$$

We can also factor this to be of the form:
$$
    \begin{array}{r@{}>{\displaystyle}l}
        p(x) = y[x_0] & {} + (x-x_0)\Biggl(y[x_1,\ x_0]                                                         \\
                      & {} + (x-x_1)\bigg(y[x_2,\ x_1,\ x_0]                                                    \\
                      & {} + (x-x_2)\Big(y[x_3,\ x_2,\ x_1,\ x_0]                                               \\
                      & {} \hspace*{2mm}\vdots                                                                  \\
                      & {} + (x-x_{n-2})\big(y[x_{n-1},\ x_{n-2},\ \cdots ,\ x_1,\ x_0]                         \\
                      & {} + (x-x_{n-1})(y[x_{n},\ x_{n-1},\ \cdots ,\ x_1,\ x_0])\big)\cdots\Big)\bigg)\Biggr) \\
    \end{array}
$$

Assuming $y[]$ has already been computed, then we only have $n-1$ additions, $n-1$ subtractions, and $n-1$ multiplications.\\
So in total, we have $3n + \mathcal{O}(1)$ flops.

% To calculate the number of flops for divided difference, we can look at the table constructed in 1 c)\\
% We can see that the first 2 cells are free to compute. For every cell after that, we have 2 additions and one division,
% which equals 3 flops per cell. So in total, we have
% $$
%     \sum_{n=1}^{n-1} n = \frac{1}{2}(n-1) (n) = \frac{n^2}{2} - \frac{n}{2}
% $$\\

% In total, we have $n-1$ subtractions, $n-1$ additions, $n-1$ multiplications, and
% $\frac{n^2}{2} - \frac{n}{2}$ flops for evaluating the divided difference. Adding all of those up, we have
% $\frac{n^2}{2} + \mathcal{O}(n)$ flops.

% $$y[x_i] = y_i \qquad \text{no flops}$$
% $$y[x_{i+k},\ \cdots\ ,\ x_i] = \frac{y[x_{i+k},\ \cdots\ ,\ x_i] - y[x_{i+k-1},\ \cdots\ ,\ x_{i+1}]}{x_{i+k} - x_i}$$

\newpage
%----------------------------------------------------------------------------------
% !                                     3
%----------------------------------------------------------------------------------
{{\LARGE \underline{\textbf{Question 3.}}}}\\
~\\\hyperlink{toc}{\hypertarget{3.1}{(a)}}\\
Suppose we are given $p(x)$ of the form $p(x) = \sum_{i=0}^{n}b_i(x-c)^i$\\
By the binomial theorem, $(x-y)^i = \sum_{k=0}^{n} \binom{n}{k} x^{n-k}(-y)^k = \sum_{k=0}^{n} \binom{n}{k} x^k(-y)^{n-k}$\\

Then\\
\scalebox{0.7}{
    \setstretch{2.5}
    $
        \begin{array}{r@{}>{\displaystyle}l}
            p(x) & {}= \sum_{i=0}^{n}b_i(x-c)^i                                                                                                                                                                                               \\
                 & {}= \sum_{i=0}^{n}b_i\sum_{k=0}^{i} \binom{i}{k} {x}^{i-k}(-c)^k                                                                                                                                                           \\
                 & {}= \sum_{i=0}^{n}\sum_{k=0}^{i}b_i \binom{i}{k} {x}^{i-k}(-c)^k                                                                                                                                                           \\
                 & {}= \sum_{k=0}^{0}b_0 \binom{0}{k} {x}^{0-k}(-c)^k                                                                                                                                                                         \\
                 & {}\qquad + \sum_{k=0}^{1}b_1 \binom{1}{k} {x}^{1-k}(-c)^k                                                                                                                                                                  \\
                 & {}\qquad + \sum_{k=0}^{2}b_2 \binom{2}{k} {x}^{2-k}(-c)^k                                                                                                                                                                  \\
                 & {}\hspace*{10.25mm} \vdots                                                                                                                                                                                                 \\
                 & {}\qquad + \sum_{k=0}^{n-1}b_{n-1} \binom{n-1}{k} {x}^{(n-1)-k}(-c)^k                                                                                                                                                      \\
                 & {}\qquad + \sum_{k=0}^{n}b_n \binom{n}{k} {x}^{n-k}(-c)^k                                                                                                                                                                  \\
                 & {}= b_0 \binom{0}{0} {x}^{0-0}(-c)^0                                                                                                                                                                                       \\
                 & {}\qquad + b_1 \binom{1}{0} {x}^{1-0}(-c)^0 + b_1 \binom{1}{1} {x}^{1-1}(-c)^1                                                                                                                                             \\
                 & {}\qquad + b_2 \binom{2}{0} {x}^{2-0}(-c)^0 + b_2 \binom{2}{1} {x}^{2-1}(-c)^1 + b_2 \binom{2}{2} {x}^{2-2}(-c)^2                                                                                                          \\
                 & {}\hspace*{10.25mm} \vdots                                                                                                                                                                                                 \\
                 & {}\qquad + b_{n-1} \binom{n-1}{0} {x}^{(n-1)-0}(-c)^0 + b_{n-1} \binom{n-1}{1} {x}^{(n-1)-1}(-c)^1 +\ \ldots\ + b_{n-1} \binom{n-1}{n-2} {x}^{(n-1)-(n-2)}(-c)^{n-2}+ b_{n-1} \binom{n-1}{n-1} {x}^{(n-1)-(n-1)}(-c)^{n-1} \\
                 & {}\qquad + b_{n} \binom{n}{0} {x}^{n-0}(-c)^0 + b_{n} \binom{n}{1} {x}^{n-1}(-c)^1 +\ \ldots\ + b_{n} \binom{n}{n-1} {x}^{n-(n-1)}(-c)^{n-1}+ b_{n-1} \binom{n}{n} {x}^{n-n}(-c)^{n}                                       \\
        \end{array}
    $
}\\

\newpage We can notice the last element of all sums are $x^0$, 2$^\text{nd}$ last element (if it has one) are $x^1$, and so on.\\
We can factor out all the $x$'s with the same power.\\

\scalebox{0.92}{
    \setstretch{2.5}
    $
        \begin{array}{r@{}>{\displaystyle}l}
            p(x) & {}= x^0\left(b_0\binom{0}{0}(-c)^0 + b_1\binom{1}{1}(-c)^1 + b_2\binom{2}{2}(-c)^2 +\ \ldots\ + b_{n-1}\binom{n-1}{n-1}(-c)^{n-1} + b_n\binom{n}{n}(-c)^n \right)                \\
                 & {}\qquad + x^1\left(b_1\binom{1}{0}(-c)^0 + b_2\binom{2}{1}(-c)^1 + b_3\binom{3}{2}(-c)^2 +\ \ldots\ + b_{n-1}\binom{n-1}{n-2}(-c)^{n-2} + b_{n}\binom{n}{n-1}(-c)^{n-1} \right) \\
                 & {}\qquad + x^2\left(b_2\binom{2}{0}(-c)^0 + b_3\binom{3}{1}(-c)^1 + b_4\binom{4}{2}(-c)^2 +\ \ldots\ + b_{n-1}\binom{n-1}{n-3}(-c)^{n-3} + b_{n}\binom{n}{n-2}(-c)^{n-2} \right) \\
                 & {}\hspace*{10.25mm} \vdots                                                                                                                                                       \\
                 & {}\qquad + x^{n-1}\left(b_{n-1}\binom{n-1}{0}(-c)^0 + b_{n}\binom{n}{1}(-c)^1\right)                                                                                             \\
                 & {}\qquad + x^{n}\left(b_{n}\binom{n}{0}(-c)^0\right)                                                                                                                             \\
        \end{array}
    $
}\\

so $$p(x) = \sum_{i=0}^{n} \left(\sum_{k=i}^{n}b_{k}\binom{k}{k-i}(-c)^{k-i}\right) x^i$$
$$\Longrightarrow p(x) = \sum_{i=0}^{n} a_ix^i \text{ where } a_i = \sum_{k=i}^{n}b_{k}\binom{k}{k-i}(-c)^{k-i}$$

\newpage\hyperlink{toc}{\hypertarget{3.2}{(b)}}\\
When calculating the reciprocal condition of the Vandermode matrix for values of $c$,
we get the following table:
\begin{center}
    \scalebox{0.8}{
        \begin{tabular}{|c|c|}
            \hline
            \texttt{c}      & Reciprocal condition     \\
            \hline
            \hline
            \texttt{0}      & \texttt{4.2535e-07}      \\
            \texttt{0.5}    & \texttt{1.9436e-06}      \\
            \texttt{1}      & \texttt{7.5962e-06}      \\
            \texttt{1.5}    & \texttt{2.6885e-05}      \\
            \texttt{2}      & \texttt{5.3226e-05}      \\
            \texttt{2.5}    & \texttt{0.0001131}       \\
            \hl{\texttt{3}} & \hl{\texttt{0.00030227}} \\
            \texttt{3.5}    & \texttt{0.00016034}      \\
            \texttt{4}      & \texttt{0.00014415}      \\
            \texttt{4.5}    & \texttt{3.5049e-05}      \\
            \texttt{5}      & \texttt{4.8742e-05}      \\
            \texttt{5.5}    & \texttt{1.9436e-06}      \\
            \texttt{6}      & \texttt{4.2535e-07}      \\
            \hline
        \end{tabular}
    }
\end{center}
We can see that for $c=3$, we get the biggest reciprocal condition $\Longrightarrow$ it minimizes the condition of the Vandermonde matrix.


To check more accurately, we can use finer values of \texttt{c}. So we have:
\begin{center}
    \scalebox{0.8}{
        \begin{tabular}{|c|c|}
            \hline
            \texttt{c}      & Reciprocal condition     \\
            \hline
            \hline
            \texttt{2.5}    & \texttt{0.0001131}       \\
            \texttt{2.55}   & \texttt{0.00011994}      \\
            \texttt{2.6}    & \texttt{0.00012769}      \\
            \texttt{2.65}   & \texttt{0.00013636}      \\
            \texttt{2.7}    & \texttt{0.00014597}      \\
            \texttt{2.75}   & \texttt{0.00015654}      \\
            \texttt{2.8}    & \texttt{0.00016809}      \\
            \texttt{2.85}   & \texttt{0.00018069}      \\
            \texttt{2.9}    & \texttt{0.00019448}      \\
            \texttt{2.95}   & \texttt{0.00020968}      \\
            \hl{\texttt{3}} & \hl{\texttt{0.00030227}} \\
            \texttt{3.05}   & \texttt{0.00028431}      \\
            \texttt{3.1}    & \texttt{0.0002672}       \\
            \texttt{3.15}   & \texttt{0.00025084}      \\
            \texttt{3.2}    & \texttt{0.00023527}      \\
            \texttt{3.25}   & \texttt{0.00022053}      \\
            \texttt{3.3}    & \texttt{0.00020671}      \\
            \texttt{3.35}   & \texttt{0.00019392}      \\
            \texttt{3.4}    & \texttt{0.00018223}      \\
            \texttt{3.45}   & \texttt{0.00017171}      \\
            \texttt{3.5}    & \texttt{0.00016034}      \\
            \hline
        \end{tabular}
    }
\end{center}
\newpage
%----------------------------------------------------------------------------------
% !                                     4
%----------------------------------------------------------------------------------
\hyperlink{toc}{\hypertarget{4}{\LARGE \underline{\textbf{Question 4.}}}}\\
\scalebox{.84}{
    \begin{tabular}{|c|c|c|c|c|c|c|c|}
        \hline
        $x_i$               & $y[x_i]$                & $y[x_{i+1},\ x_i]$                            & $y[x_{i+2},\ x_{i+1},\ x_i]$                         & $y[x_{i+3},\ \cdots\ ,\ x_i]$                  & $y[x_{i+4},\ \cdots\ ,\ x_i]$                  & $y[x_{i+5},\ \cdots\ ,\ x_i]$                  & $y[x_{i+6},\ \cdots\ ,\ x_i]$                 \\\hline\hline
        \multirow{2}{*}{-1} & \multirow{2}{*}{\hl{4}} & \multicolumn{1}{c}{}                          & \multicolumn{1}{c}{}                                 & \multicolumn{1}{c}{}                           & \multicolumn{1}{c}{}                           & \multicolumn{1}{c}{}                           &                                               \\\cline{3-3}
                            &                         & \multirow{2}{*}{$\frac{7-4}{0-(-1)}=$ \hl{3}} & \multicolumn{1}{c}{}                                 & \multicolumn{1}{c}{}                           & \multicolumn{1}{c}{}                           & \multicolumn{1}{c}{}                           &                                               \\\cline{0-1}\cline{4-4}
        \multirow{2}{*}{0}  & \multirow{2}{*}{7}      &                                               & \multirow{2}{*}{$\frac{6-3}{0-(-1)}=$ \hl{3}}        & \multicolumn{1}{c}{}                           & \multicolumn{1}{c}{}                           & \multicolumn{1}{c}{}                           &                                               \\\cline{3-3}\cline{5-5}
                            &                         & \multirow{2}{*}{$\frac{y^\prime(0)}{1!}=6$}   &                                                      & \multirow{2}{*}{$\frac{15-3}{1-(-1)}=$ \hl{6}} & \multicolumn{1}{c}{}                           & \multicolumn{1}{c}{}                           &                                               \\\cline{0-1}\cline{4-4}\cline{6-6}
        \multirow{2}{*}{0}  & \multirow{2}{*}{7}      &                                               & \multirow{2}{*}{$\frac{21-6}{1-0}=15$}               &                                                & \multirow{2}{*}{$\frac{20-6}{1-(-1)}=$ \hl{7}} & \multicolumn{1}{c}{}                           &                                               \\\cline{3-3}\cline{5-5}\cline{7-7}
                            &                         & \multirow{2}{*}{$\frac{28-7}{1-0}=21$}        &                                                      & \multirow{2}{*}{$\frac{35-15}{1-0}=20$}        &                                                & \multirow{2}{*}{$\frac{15-7}{1-(-1)}=$ \hl{4}} &                                               \\\cline{0-1}\cline{4-4}\cline{6-6}\cline{8-8}
        \multirow{2}{*}{1}  & \multirow{2}{*}{28}     &                                               & \multirow{2}{*}{$\frac{56-21}{1-0}= 35$}             &                                                & \multirow{2}{*}{$\frac{35-20}{1-0}=15$}        &                                                & \multirow{2}{*}{$\frac{7-4}{2-(-1)}=$ \hl{1}} \\\cline{3-3}\cline{5-5}\cline{7-7}
                            &                         & \multirow{2}{*}{$\frac{y^\prime(1)}{1!}=56$}  &                                                      & \multirow{2}{*}{$\frac{70-35}{1-0}=35$}        &                                                & \multirow{2}{*}{$\frac{29-15}{2-0}=7$}         &                                               \\\cline{0-1}\cline{4-4}\cline{6-6}\cline{8-8}
        \multirow{2}{*}{1}  & \multirow{2}{*}{28}     &                                               & \multirow{2}{*}{$\frac{y^{\prime\prime}(1)}{2!}=70$} &                                                & \multirow{2}{*}{$\frac{93-35}{2-0}=29$}        &                                                &                                               \\\cline{3-3}\cline{5-5}\cline{7-7}
                            &                         & \multirow{2}{*}{$\frac{y^\prime(1)}{1!}=56$}  &                                                      & \multirow{2}{*}{$\frac{163-70}{2-1}=93$}       &                                                & \multicolumn{1}{c}{}                           &                                               \\\cline{0-1}\cline{4-4}\cline{6-6}
        \multirow{2}{*}{1}  & \multirow{2}{*}{28}     &                                               & \multirow{2}{*}{$\frac{219-56}{2-1}=163$}            &                                                & \multicolumn{1}{c}{}                           & \multicolumn{1}{c}{}                           &                                               \\\cline{3-3}\cline{5-5}
                            &                         & \multirow{2}{*}{$\frac{247-28}{2-1}=219$}     &                                                      & \multicolumn{1}{c}{}                           & \multicolumn{1}{c}{}                           & \multicolumn{1}{c}{}                           &                                               \\\cline{0-1}\cline{4-4}
        \multirow{2}{*}{2}  & \multirow{2}{*}{247}    &                                               & \multicolumn{1}{c}{}                                 & \multicolumn{1}{c}{}                           & \multicolumn{1}{c}{}                           & \multicolumn{1}{c}{}                           &                                               \\\cline{3-3}
                            &                         & \multicolumn{1}{c}{}                          & \multicolumn{1}{c}{}                                 & \multicolumn{1}{c}{}                           & \multicolumn{1}{c}{}                           & \multicolumn{1}{c}{}                           &                                               \\\hline
    \end{tabular}
}

$$
    \begin{array}{r@{}>{\displaystyle}l}
        p(x) & {}= y[x_0] + (x-x_0)y[x_1,\ x_0]                                                                      \\
             & {} \hspace*{14.25mm} + (x-x_0)(x-x_1)y[x_2,\ x_1,\ x_0]                                               \\
             & {} \hspace*{14.25mm} + (x-x_0)(x-x_1)(x-x_2)y[x_3,\ x_2,\ x_1,\ x_0]                                  \\
             & {} \hspace*{14.25mm} + (x-x_0)(x-x_1)(x-x_2)(x-x_3)y[x_4,\ x_3,\ x_2,\ x_1,\ x_0]                     \\
             & {} \hspace*{14.25mm} + (x-x_0)(x-x_1)(x-x_2)(x-x_3)(x-x_4)y[x_5,\ x_3,\ x_2,\ x_1,\ x_0]              \\
             & {} \hspace*{14.25mm} + (x-x_0)(x-x_1)(x-x_2)(x-x_3)(x-x_4)(x-x_5)y[x_6,\ x_5,\ x_3,\ x_2,\ x_1,\ x_0] \\
             & {}=              4 + \big(x-(-1)\big)(3)                                                              \\
             & {} \hspace*{7.8mm} + \big(x-(-1)\big)(x)(3)                                                           \\
             & {} \hspace*{7.8mm} + \big(x-(-1)\big)(x)(x)(6)                                                        \\
             & {} \hspace*{7.8mm} + \big(x-(-1)\big)(x)(x)(x-1)(7)                                                   \\
             & {} \hspace*{7.8mm} + \big(x-(-1)\big)(x)(x)(x-1)(x-1)(4)                                              \\
             & {} \hspace*{7.8mm} + \big(x-(-1)\big)(x)(x)(x-1)(x-1)(x-1)(1)                                         \\
             & {}= 4 + 3(x+1) + 6x^2(x+1) + 7x^2(x+1)(x-1) + 4x^2(x+1)(x-1)^2 + x^2(x+1)(x-1)^3                      \\
             & {}= x^6 + 2 x^5 + 3 x^4 + 4 x^3 + 5 x^2 + 6 x + 7
    \end{array}
$$
$\hspace*{0.8mm}p^\prime(x) = 6 x^5 + 10 x^4 + 12 x^3 + 12 x^2  + 10 x + 6  $\\
$p^{\prime\prime}(x) = 30 x^4 + 40 x^3+ 36 x^2 + 24 x + 10  $\\
$$
    \begin{matrix}
        p(-1) = 0 & p(0) = 7        & p(1) = 28                 & p(2) = 247 \\
                  & p^\prime(0) = 6 & p^\prime(1) = 56          &            \\
                  &                 & p^{\prime\prime}(1) = 140 &            \\
    \end{matrix}
$$
\end{document}
