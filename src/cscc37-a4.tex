\documentclass[12pt]{article}

\usepackage[margin=1in, left=0.6in, right=0.6in]{geometry}
\usepackage{fancyhdr} % header
\usepackage{hyperref} % links

\usepackage{amsmath,amsthm,amssymb}	%math stuff
\usepackage{graphicx} \graphicspath{ {./images/} }
\usepackage{setspace} % increase line spacing
\usepackage{tabularx} % long tables
\usepackage{enumitem} % labelling itmes
\usepackage{color, soul}
\usepackage{lmodern} % bolding \texttt{}
\usepackage[T1]{fontenc} % for {} in \texttt{}
\usepackage{listings}
\usepackage[table]{xcolor}
\usepackage[edges]{forest}
\usepackage{xfrac} % slanted fractions
\usepackage{changepage}   % for the adjustwidth environment
% \usepackage{array}
% \usepackage{booktabs}
% \usepackage{siunitx}
% \usepackage{alltt}

\definecolor{dkgreen}{rgb}{0,0.6,0}
\definecolor{gray}{rgb}{0.5,0.5,0.5}
\definecolor{mauve}{rgb}{0.58,0,0.82}
\definecolor{backcolour}{rgb}{0.95,0.95,0.92}

\setlength{\parindent}{0pt}

\pagestyle{fancy}
\fancyhead[LO,L]{CSCC37 A4}
\fancyhead[CO,C]{Stephen Guo}
\fancyhead[RO,R]{1006313231}
\fancyfoot[LO,L]{}
\fancyfoot[CO,C]{\thepage}
\fancyfoot[RO,R]{}

\newcommand{\N}{\mathbb{N}}
\newcommand{\R}{\mathbb{R}}
\newcommand{\Rplus}{\mathbb{R}^{+}}
\newcommand{\bigbracket}[1]{\big(#1\big)}
\newcommand{\Bigbracket}[1]{\Big(#1\Big)}
\newcommand{\floorSurround}[1]{\left\lfloor#1\right\rfloor}
\newcommand{\ceilingSurround}[1]{\left\lceil#1\right\rceil}
\newcommand{\code}[1]{{\ttfamily \fontseries{b}\selectfont #1}}
\definecolor{codegray}{gray}{0.9}
\def \calO {\mathcal{O}}
\newcommand{\bigO}[1]{\ensuremath{\calO(#1)}}
\renewcommand{\qed}{\hfill$\blacksquare$}
\newenvironment{proofindent}{\vspace*{2mm}\hfill\begin{minipage}{\dimexpr\textwidth-10mm}}{\end{minipage}}

\everymath{\displaystyle}

\begin{document}
%----------------------------------------------------------------------------------
%                              Table of Contents
%----------------------------------------------------------------------------------
\begin{center}
    \hypertarget{toc}{\LARGE \underline{\textbf{Table of Contents}}}\\
\end{center}

{\textbf{Question 1:}}
\vspace{1mm}
\hrule
\vspace{1mm}
\hyperlink{1.1}{(a)}\\
\hyperlink{1.2}{(b)}\\
\hyperlink{1.3}{(c)}\\
\hyperlink{1.4}{(d)}\\
\hyperlink{1.5}{(e)}\\
\hyperlink{1.6}{(f)}\\

\textbf{Question 2:}
\vspace{1mm}
\hrule
\vspace{1mm}
\hyperlink{2.1}{(a)}\\
\hyperlink{2.2}{(b)}\\

\textbf{Question 3:}
\vspace{1mm}
\hrule
\vspace{1mm}
\hyperlink{3.1}{(a)}\\
\hyperlink{3.2}{(b)}\\

\hyperlink{4}{\textbf{Question 4:}}
\vspace{1mm}
\hrule
\vspace{1mm} \leavevmode \\

\newpage

% {\setstretch{1.5}$\begin{array}{r@{}>{\displaystyle}l}  \end{array}$}
% {
%     \setstretch{1.5}
%     $
%         \begin{array}{r@{}>{\displaystyle}l}
%              & {} \\
%              & {} \\
%              & {} \\
%              & {} \\
%              & {} \\
%         \end{array}
%     $
% }
%----------------------------------------------------------------------------------
%                                   Questions
%----------------------------------------------------------------------------------
\setstretch{1.2}
%----------------------------------------------------------------------------------
% !                                     1
%----------------------------------------------------------------------------------
{\LARGE\underline{\textbf{Question 1.}}}\\
~\\\hyperlink{toc}{\hypertarget{1.1}{(a)}}\\
{
\setstretch{1.5}
$
    \begin{array}{r@{}>{\displaystyle}l}
        p(-1) & {}= {4} \\
        p(0)  & {}={6}  \\
        p(1)  & {}={12} \\
    \end{array}
$
}
$\Longleftrightarrow$
$
    \begin{bmatrix}
        1 & -1 & 1 \\
        1 & 0  & 0 \\
        1 & 1  & 1 \\
    \end{bmatrix}
    \begin{bmatrix}
        a_0 \\a_1\\a_2
    \end{bmatrix}
    =
    \begin{bmatrix}
        4 \\6\\12
    \end{bmatrix}
$\\

Eliminate 1$^{\text{st}}$ column:\\
\begin{minipage}[t]{0.5\textwidth}
    $$
        P_1 = I =
        \begin{bmatrix}
            1 & 0 & 0 \\
            0 & 1 & 0 \\
            0 & 0 & 1 \\
        \end{bmatrix}
    $$
\end{minipage}
\begin{minipage}[t]{0.5\textwidth}
    $$
        P_1 A= A =
        \begin{bmatrix}
            1 & -1 & 1 \\
            1 & 0  & 0 \\
            1 & 1  & 1 \\
        \end{bmatrix}
    $$
\end{minipage}\\

\begin{minipage}[t]{0.5\textwidth}
    $$
        L_1 =
        \begin{bmatrix}
            1  & 0 & 0 \\
            -1 & 1 & 0 \\
            -1 & 0 & 1 \\
        \end{bmatrix}
    $$
\end{minipage}
\begin{minipage}[t]{0.5\textwidth}
    $$
        L_1P_1A =
        \begin{bmatrix}
            1 & -1 & 1  \\
            0 & 1  & -1 \\
            0 & 2  & 0  \\
        \end{bmatrix}
    $$
\end{minipage}\\

Eliminate 2$^{\text{nd}}$ column:\\
\begin{minipage}[t]{0.5\textwidth}
    $$
        P_2 =
        \begin{bmatrix}
            1 & 0 & 0 \\
            0 & 0 & 1 \\
            0 & 1 & 0 \\
        \end{bmatrix}
    $$
\end{minipage}
\begin{minipage}[t]{0.5\textwidth}
    $$
        P_2 L_1P_1A =
        \begin{bmatrix}
            1 & -1 & 1  \\
            0 & 2  & 0  \\
            0 & 1  & -1 \\
        \end{bmatrix}
    $$
\end{minipage}\\
\begin{minipage}[t]{0.5\textwidth}
    $$
        L_2 =
        \begin{bmatrix}
            1 & 0             & 0 \\
            0 & 1             & 0 \\
            0 & -\sfrac{1}{2} & 1 \\
        \end{bmatrix}
    $$
\end{minipage}
\begin{minipage}[t]{0.5\textwidth}
    $$
        L_2 P_2 L_1P_1A =
        \begin{bmatrix}
            1 & -1 & 1  \\
            0 & 2  & 0  \\
            0 & 0  & -1 \\
        \end{bmatrix}
    $$
\end{minipage}\\

{
\setstretch{1.5}
$
    \begin{array}{rrr@{}l}
        L_2 P_2 L_1P_1A = U
         & \Longleftrightarrow & L_2 P_2 L_1P_2P_2P_1A      & {}= U                               \\
         & \Longleftrightarrow & L_2 (P_2 L_1P_2)P_2P_1A    & {}= U                               \\
         & \Longleftrightarrow & L_2 \widetilde{L_1}P_2P_1A & {}= U                               \\
         & \Longleftrightarrow & P_2P_1A                    & {}= \widetilde{L_1}^{-1} L_2^{-1} U \\
         & \Longleftrightarrow & PA                         & {}= L U                             \\
    \end{array}
$
}\\

\begin{minipage}[t]{0.5\textwidth}
    $
        \widetilde{L_1} = P_2 L_1P_2 =
        \begin{bmatrix}
            1  & 0 & 0 \\
            -1 & 1 & 0 \\
            -1 & 0 & 1 \\
        \end{bmatrix}
    $
\end{minipage}
\begin{minipage}[t]{0.5\textwidth}
    $
        L = \widetilde{L_1}^{-1} L_2^{-1} =
        \begin{bmatrix}
            1 & 0            & 0 \\
            1 & 1            & 0 \\
            1 & \sfrac{1}{2} & 1 \\
        \end{bmatrix}
    $
\end{minipage}\\





$
    P = P_2P_1 =
    \begin{bmatrix}
        1 & 0 & 0 \\
        0 & 0 & 1 \\
        0 & 1 & 0 \\
    \end{bmatrix}
$\\

$$
    \begin{bmatrix}
        1 & 0 & 0 \\
        0 & 0 & 1 \\
        0 & 1 & 0 \\
    \end{bmatrix}
    \begin{bmatrix}
        1 & -1 & 1 \\
        1 & 0  & 0 \\
        1 & 1  & 1 \\
    \end{bmatrix}
    =
    \begin{bmatrix}
        1 & 0            & 0 \\
        1 & 1            & 0 \\
        1 & \sfrac{1}{2} & 1 \\
    \end{bmatrix}
    \begin{bmatrix}
        1 & -1 & 1  \\
        0 & 2  & 0  \\
        0 & 0  & -1 \\
    \end{bmatrix}
$$\\

$$
    A\vec{x} = \vec{b}
    \Longleftrightarrow PA\vec{x} = P\vec{b}
    \Longleftrightarrow LU\vec{x} = P\vec{b}
    \Longleftrightarrow L(U\vec{x}) = P\vec{b}
    \Longleftrightarrow L\vec{d} = P\vec{b}
$$

$
    P\vec{b} = \begin{bmatrix}
        4 \\12\\6
    \end{bmatrix}
$\\
Forward solve $L\vec{d} = \vec{b}$ for $\vec{d}$:\\
$
    \begin{bmatrix}
        1 & 0            & 0 \\
        1 & 1            & 0 \\
        1 & \sfrac{1}{2} & 1 \\
    \end{bmatrix}
    \begin{bmatrix}
        d_1 \\d_2\\d_3
    \end{bmatrix}
    =
    \begin{bmatrix}
        4 \\12\\6
    \end{bmatrix}
$\\
$d_1 = 4$\\
$d_1 + d_2 = 12$\\
$d_1 + \frac{1}{2}d_2 + d_3 = 6$\\

$d_1 = 4$\\
$d_2 = 8$\\
$d_3 = -2$\\

Backward solve $U\vec{x} = \vec{d}$ for $\vec{x}$\\
$
    \begin{bmatrix}
        1 & -1 & 1  \\
        0 & 2  & 0  \\
        0 & 0  & -1 \\
    \end{bmatrix}
    \begin{bmatrix}
        x_1 \\x_2\\x_3
    \end{bmatrix}
    =
    \begin{bmatrix}
        4 \\8\\-2
    \end{bmatrix}
$\\

$x_3 = 2$\\
$x_2 = 4$\\
$x_1 = 6$\\

$\therefore p(x) = 6 + 4x + 2x^2$
\newpage\hyperlink{toc}{\hypertarget{1.2}{(b)}}\\
$$
    \displaystyle l_i(x) = \prod_{\substack{j=0\\j\not=i}}^{n}\frac{x-x_j}{x_i-x_j}
    \qquad i =0,\ 1,\ 2
$$\\
{
\setstretch{1.5}
$
    \begin{array}{r@{}>{\displaystyle}l}
        x_{0} & {}= -1 \\
        x_{1} & {}= 0  \\
        x_{2} & {}= 1  \\
    \end{array}
$
}
{
\setstretch{1.5}
$
    \begin{array}{r@{}>{\displaystyle}l}
        y_{0} & {}= 4  \\
        y_{1} & {}= 6  \\
        y_{2} & {}= 12 \\
    \end{array}
$
}\\

\begin{minipage}[t]{0.3333333333333333333333333\textwidth}
    \begin{center}
        $i=0$
    \end{center}
    $j = 1,\ 2$\\
    {
    \setstretch{2.5}
    $
        \begin{array}{r@{}>{\displaystyle}l}
            l_{0}(x) & {}= \left(\frac{x-0}{-1-0}\right)\left(\frac{x-1}{-1-1}\right) \\
                     & {}= \left(-x\right)\left(\frac{x-1}{-2}\right)                 \\
                     & {}= \frac{1}{2} x(x-1)                                         \\
        \end{array}
    $
    }
\end{minipage}
\begin{minipage}[t]{0.3333333333333333333333333\textwidth}
    \begin{center}
        $i=1$
    \end{center}
    $j = 0,\ 2$\\
    {
    \setstretch{2.5}
    $
        \begin{array}{r@{}>{\displaystyle}l}
            l_{1}(x) & {}= \left(\frac{x-(-1)}{0-(-1)}\right)\left(\frac{x-1}{0-1}\right) \\
                     & {}= (x+1)(1-x)                                                     \\
        \end{array}
    $
    }
\end{minipage}
\begin{minipage}[t]{0.3333333333333333333333333\textwidth}
    \begin{center}
        $i=2$
    \end{center}
    $j = 0,\ 1$\\
    {
    \setstretch{2.5}
    $
        \begin{array}{r@{}>{\displaystyle}l}
            l_{2}(x) & {}= \left(\frac{x-(-1)}{1-(-1)}\right)\left(\frac{x-0}{1-0}\right) \\
                     & {}= \left(\frac{x+1}{2}\right)(x)                                  \\
                     & {}=\frac{1}{2}x(x+1)                                               \\
        \end{array}
    $
    }
\end{minipage}

{
\setstretch{2.5}
$$
    \begin{array}{r@{}>{\displaystyle}l}
        p(x) & {}= \sum^{n}_{i=0}l_i(x)y_i                                       \\
             & {}= l_0(x)y_0 + l_1(x)y_1 + l_2(x)y_2                             \\
             & {}= \frac{1}{2} x(x-1)(4) + (x+1)(1-x)(6) + \frac{1}{2}x(x+1)(12) \\
             & {}= 2 x^2-2x + (-6x^2) + 6 + 6x^2 + 6x                            \\
             & {}= 2 x^2+4x + 6                                                  \\
    \end{array}
$$
}

~\\\hyperlink{toc}{\hypertarget{1.3}{(c)}}\\
~\\\hyperlink{toc}{\hypertarget{1.4}{(d)}}\\
~\\\hyperlink{toc}{\hypertarget{1.5}{(e)}}\\
~\\\hyperlink{toc}{\hypertarget{1.6}{(f)}}\\
\newpage
%----------------------------------------------------------------------------------
% !                                     2
%----------------------------------------------------------------------------------
{\LARGE \underline{\textbf{Question 2.}}}\\
~\\\hyperlink{toc}{\hypertarget{2.1}{(a)}}\\
~\\\hyperlink{toc}{\hypertarget{2.2}{(b)}}\\
\newpage
%----------------------------------------------------------------------------------
% !                                     3
%----------------------------------------------------------------------------------
{{\LARGE \underline{\textbf{Question 3.}}}}\\
~\\\hyperlink{toc}{\hypertarget{3.1}{(a)}}\\
~\\\hyperlink{toc}{\hypertarget{3.2}{(b)}}\\
\newpage
%----------------------------------------------------------------------------------
% !                                     4
%----------------------------------------------------------------------------------
\hyperlink{toc}{\hypertarget{4}{\LARGE \underline{\textbf{Question 4.}}}}\\
\end{document}
