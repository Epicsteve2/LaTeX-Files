\documentclass[12pt]{article}

\usepackage[margin=1in, left=0.6in, right=0.6in]{geometry}
\usepackage{fancyhdr}	% header
\usepackage{hyperref} % links

\usepackage{amsmath,amsthm,amssymb}	%math stuff
\usepackage{graphicx} \graphicspath{ {./images/} }
\usepackage{setspace} % increase line spacing
\usepackage{tabularx} % long tables
\usepackage{enumitem} % labelling itmes
\usepackage{color, soul}
\usepackage{lmodern} % bolding \texttt{}
\usepackage[T1]{fontenc} % for {} in \texttt{}
\usepackage{listings}
\usepackage[table]{xcolor}
\usepackage[edges]{forest}
\usepackage{xfrac} % slanted fractions
% \usepackage{array}
% \usepackage{booktabs}
% \usepackage{siunitx}
% \usepackage{alltt}

\definecolor{dkgreen}{rgb}{0,0.6,0}
\definecolor{gray}{rgb}{0.5,0.5,0.5}
\definecolor{mauve}{rgb}{0.58,0,0.82}
\definecolor{backcolour}{rgb}{0.95,0.95,0.92}

\setlength{\parindent}{0pt}

\pagestyle{fancy}
\fancyhead[LO,L]{CSCC37 A2}
\fancyhead[CO,C]{Stephen Guo}
\fancyhead[RO,R]{1006313231}
\fancyfoot[LO,L]{}
\fancyfoot[CO,C]{\thepage}
\fancyfoot[RO,R]{}

\newcommand{\N}{\mathbb{N}}
\newcommand{\R}{\mathbb{R}}
\newcommand{\Rplus}{\mathbb{R}^{+}}
\newcommand{\bigbracket}[1]{\big(#1\big)}
\newcommand{\Bigbracket}[1]{\Big(#1\Big)}
\newcommand{\floorSurround}[1]{\left\lfloor#1\right\rfloor}
\newcommand{\ceilingSurround}[1]{\left\lceil#1\right\rceil}
\newcommand{\code}[1]{{\ttfamily \fontseries{b}\selectfont #1}}
\definecolor{codegray}{gray}{0.9}
\def \calO {\mathcal{O}}
\newcommand{\bigO}[1]{\ensuremath{\calO(#1)}}
\renewcommand{\qed}{\hfill$\blacksquare$}
\newenvironment{proofindent}{\vspace*{2mm}\hfill\begin{minipage}{\dimexpr\textwidth-10mm}}{\end{minipage}}

\everymath{\displaystyle}

\begin{document}
%----------------------------------------------------------------------------------
%                              Table of Contents
%----------------------------------------------------------------------------------
\begin{center}
    \hypertarget{toc}{\LARGE \noindent \underline{\textbf{Table of Contents}}}\\
\end{center}

\noindent {\textbf{Question 1:}}
\vspace{1mm}
\hrule
\vspace{1mm}
\noindent
\hyperlink{1.1}{(a)}\\
\hyperlink{1.2}{(b)}\\
\hyperlink{1.3}{(c)}\\
\hyperlink{1.4}{(d)}\\

\noindent \hyperlink{2}{\textbf{Question 2:}}
\vspace{1mm}
\hrule
\vspace{1mm} \leavevmode \\

\noindent \hyperlink{3}{\textbf{Question 3:}}
\vspace{1mm}
\hrule
\vspace{1mm}
\noindent
\hyperlink{3.1}{(a)}\\
\hyperlink{3.2}{(b)}\\
\hyperlink{3.3}{(c)}\\
\hyperlink{3.4}{(d)}\\

\noindent \hyperlink{4}{\textbf{Question 4:}}
\vspace{1mm}
\hrule
\vspace{1mm} \leavevmode \\

\newpage

%{\setstretch{1.5}$\begin{array}{r@{}>{\displaystyle}l}  \end{array}$}
% {\setstretch{1.5}$\begin{array}{r@{}>{\displaystyle}l}
% 	&{} \\
% 	&{} \\
% 	&{} \\
% 	&{} \\
% 	&{} \\
% \end{array}$}
%----------------------------------------------------------------------------------
%                                   Questions
%----------------------------------------------------------------------------------
\setstretch{1.2}
%----------------------------------------------------------------------------------
% !                                    1
%----------------------------------------------------------------------------------
\noindent \hyperlink{toc}{\LARGE \underline{\textbf{Question 1.}}}\\\\
\noindent \hyperlink{toc}{\hypertarget{1.1}{(a)}}\\
Let $\mathcal{L}_k$ be a Gauss Transform. Then it has the form\\
$
    \mathcal{L}_k =
    \begin{bmatrix}
        1      & 0      & \cdots & 0      & 0       & 0      & \cdots & 0      \\
        0      & 1      & \cdots & 0      & 0       & 0      & \cdots & 0      \\
        \vdots & \vdots & \ddots & \vdots & \vdots  & \vdots & \ddots & \vdots \\
        0      & 0      & \cdots & 1      & 0       & 0      & \cdots & 0      \\
        0      & 0      & \cdots & 0      & 1       & 0      & \cdots & 0      \\
        0      & 0      & \cdots & 0      & a_{k+1} & 1      & \cdots & 0      \\
        \vdots & \vdots & \ddots & \vdots & \vdots  & \vdots & \ddots & \vdots \\
        0      & 0      & \cdots & 0      & a_{n}   & 0      & \cdots & 1      \\
    \end{bmatrix}
$ \hspace*{3cm}
Then let $m_k =
    \begin{bmatrix}
        0        \\
        0        \\
        \vdots   \\
        0        \\
        0        \\
        -a_{k+1} \\
        \vdots   \\
        -a_{n}   \\
    \end{bmatrix}$\\[1cm]

Now $-m_k {e_k}^T =
    \begin{bmatrix}
        0      & 0      & \cdots & 0      & 0       & 0      & \cdots & 0      \\
        0      & 0      & \cdots & 0      & 0       & 0      & \cdots & 0      \\
        \vdots & \vdots & \ddots & \vdots & \vdots  & \vdots & \ddots & \vdots \\
        0      & 0      & \cdots & 0      & 0       & 0      & \cdots & 0      \\
        0      & 0      & \cdots & 0      & 0       & 0      & \cdots & 0      \\
        0      & 0      & \cdots & 0      & a_{k+1} & 0      & \cdots & 0      \\
        \vdots & \vdots & \ddots & \vdots & \vdots  & \vdots & \ddots & \vdots \\
        0      & 0      & \cdots & 0      & a_{n}   & 0      & \cdots & 0      \\
    \end{bmatrix}
$\\[1cm]

So $\mathcal{L}_k = I - m_k {e_k}^T$

\newpage
\noindent \hyperlink{toc}{\hypertarget{1.2}{(b)}}\\
By definition, the inverse of a Gauss Transform is simply flipping the signs of the non-identity values. \\
So if we invert the signs of $-m_k$, we have
$$m_k =
    \begin{bmatrix}
        0       \\
        0       \\
        \vdots  \\
        0       \\
        0       \\
        a_{k+1} \\
        \vdots  \\
        a_{n}   \\
    \end{bmatrix}
$$
Then $$I + m_k {e_k}^T =
    \begin{bmatrix}
        1      & 0      & \cdots & 0      & 0        & 0      & \cdots & 0      \\
        0      & 1      & \cdots & 0      & 0        & 0      & \cdots & 0      \\
        \vdots & \vdots & \ddots & \vdots & \vdots   & \vdots & \ddots & \vdots \\
        0      & 0      & \cdots & 1      & 0        & 0      & \cdots & 0      \\
        0      & 0      & \cdots & 0      & 1        & 0      & \cdots & 0      \\
        0      & 0      & \cdots & 0      & -a_{k+1} & 1      & \cdots & 0      \\
        \vdots & \vdots & \ddots & \vdots & \vdots   & \vdots & \ddots & \vdots \\
        0      & 0      & \cdots & 0      & -a_{n}   & 0      & \cdots & 1      \\
    \end{bmatrix}
    = {\mathcal{L}_k}^{-1}$$\newpage

\noindent \hyperlink{toc}{\hypertarget{1.3}{(c)}}\\
WTS: $(L_kL_j)^{-1} = (I + m_k {e_k}^T) + (m_j {e_j}^T) - I$\\

{
\setstretch{1.5}
$
    \begin{array}{r@{}>{\displaystyle}ll}
        LHS
         & {}= (L_kL_j)^{-1}                                                           & [\text{given}]                             \\
         & {}= {L_j}^{-1} {L_k}^{-1}                                                   & [\text{by inverse laws}]                   \\
         & {}= (I + m_j {e_j}^T)(I + m_k {e_k}^T)                                      & [\text{by (b)}]                            \\
         & {}= I + m_j {e_j}^T + m_k {e_k}^T + m_j {e_j}^T m_k {e_k}^T \hspace*{1.5cm} & [\text{expanding}]                         \\
         & {}= I + m_j {e_j}^T + m_k {e_k}^T + 0                                       & [\text{since $j<k$ and ${e_j}^T m_k = 0$}] \\
         & {}= I + m_j {e_j}^T + m_k {e_k}^T + I - I                                   & [I - I = 0]                                \\
         & {}= (I + m_k {e_k}^T) + (I + m_j {e_j}^T) - I                               & [\text{rearranging}]                       \\
         & {}= RHS                                                                     & [\text{given}]                             \\
    \end{array}
$\\
as wanted
}
\\[5mm]

\newpage
\noindent \hyperlink{toc}{\hypertarget{1.4}{(d)}}\\
WTS: $\widetilde{\mathcal{L}_k} = \mathcal{L}_k$ with multipliers $i$ and $j$ swapped\\

{
\setstretch{1.5}
$
    \begin{array}{r@{}>{\displaystyle}ll}
        LHS
         & {}= \widetilde{\mathcal{L}_k}                                                  & [\text{given}]                                      \\
         & {}= P_i \mathcal{L}_k P_i                                                      & [\text{given}]                                      \\
         & {}= P_i (I - m_k {e_k}^T) P_i                                                  & [\text{by (a)}]                                     \\
         & {}= (P_i - P_i m_k {e_k}^T) P_i                                                & [\text{expanding}]                                  \\
         & {}= P_i P_i - P_i m_k {e_k}^T P_i                                              & [\text{expanding}]                                  \\
         & {}= I - P_i m_k {e_k}^T P_i                                                    & [\text{since $P_i P_i = I$}]                        \\
         & {}= \mathcal{L}_k \text{ with multipliers $i$ and $j$ swapped} \hspace*{1.5cm} & [\text{since $m_k {e_k}^T$ multiplers got swapped}] \\
         & {}= RHS                                                                        & [\text{given}]                                      \\
    \end{array}
$\\
as wanted
}
\\[5mm]

\newpage
%----------------------------------------------------------------------------------
% !                                    2
%----------------------------------------------------------------------------------
\noindent \hyperlink{toc}{\hypertarget{2}{\LARGE \underline{\textbf{Question 2.}}}}\\\\
% I GET IT. You put the equal sign in its own column to "align" it!!!
{\setstretch{1.5}$
    \begin{array}{rrr@{}>{\displaystyle}ll}
        PA=LU
         & \Longleftrightarrow & \det(PA)       & {}= \det(LU)                      & [\text{det both sides}]      \\
         & \Longleftrightarrow & \det(P)\det(A) & {}= \det(L)\det(U) \hspace*{15mm} & [\text{by det laws}]         \\
         & \Longleftrightarrow & \det(A)        & {}= \det(L)\det(U)                & [\text{since $\det(P) = 1$}] \\
    \end{array}$
}\\[1cm]
This is much more efficient than just finding the determinant of $A$ since the determinant
of a triangular matrix is just the product of the diagonal.
\newpage
%----------------------------------------------------------------------------------
% !                                    3
%----------------------------------------------------------------------------------
\noindent \hyperlink{toc}{{\LARGE \underline{\textbf{Question 3.}}}}\\\\
\noindent \hyperlink{toc}{\hypertarget{3.1}{(a)}}\\
$$
    A=
    \begin{bmatrix}
        3 & 3 & 9 & 6 \\
        4 & 4 & 4 & 4 \\
        1 & 1 & 5 & 5 \\
        2 & 2 & 4 & 6 \\
    \end{bmatrix},\hspace*{1cm}
    b=
    \begin{bmatrix}
        21 \\
        24 \\
        10 \\
        16 \\
    \end{bmatrix}
$$

Eliminate $1^\text{st}$ column:\\
$$
    P_1 = P_{12} =
    \begin{bmatrix}
        0 & 1 & 0 & 0 \\
        1 & 0 & 0 & 0 \\
        0 & 0 & 1 & 0 \\
        0 & 0 & 0 & 1 \\
    \end{bmatrix},\hspace*{1cm}
    P_1A=
    \begin{bmatrix}
        4 & 4 & 4 & 4 \\
        3 & 3 & 9 & 6 \\
        1 & 1 & 5 & 5 \\
        2 & 2 & 4 & 6 \\
    \end{bmatrix},\hspace*{1cm}
    L_1=
    \begin{bmatrix}
        1             & 0 & 0 & 0 \\
        \sfrac{-3}{4} & 1 & 0 & 0 \\
        \sfrac{-1}{4} & 0 & 1 & 0 \\
        \sfrac{-1}{2} & 0 & 0 & 1 \\
    \end{bmatrix}
$$

$$
    \begin{bmatrix}
         &  &  & \\
         &  &  & \\
         &  &  & \\
         &  &  & \\
    \end{bmatrix}
    \begin{bmatrix}
        1 & 0 & 0 & 0 \\
        0 & 1 & 0 & 0 \\
        0 & 0 & 1 & 0 \\
        0 & 0 & 0 & 1 \\
    \end{bmatrix}
$$

\noindent ~\\\hyperlink{toc}{\hypertarget{3.2}{(b)}}\\
\noindent ~\\\hyperlink{toc}{\hypertarget{3.3}{(c)}}\\
\noindent ~\\\hyperlink{toc}{\hypertarget{3.4}{(d)}}\\

\newpage
%----------------------------------------------------------------------------------
% !                                    4
%----------------------------------------------------------------------------------
\noindent \hyperlink{toc}{\hypertarget{4}{\LARGE \underline{\textbf{Question 4.}}}}\\\\
\end{document}