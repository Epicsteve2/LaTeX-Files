\documentclass[12pt]{article}
	
\usepackage[margin=1in]{geometry}	
\usepackage{amsmath,amsthm,amssymb,scrextend}			
\usepackage{fancyhdr}				
\usepackage{graphicx}				
\usepackage{cancel}					
\usepackage{changepage}
\usepackage{color,soul}
\usepackage{enumitem}
\usepackage{array}
\usepackage{hyperref}
\usepackage{setspace}
\usepackage{tabularx}
\usepackage{pifont}

\pagestyle{fancy}
\fancyhead[LO,L]{MATB24 Homework 5}
\fancyhead[CO,C]{Stephen Guo}
\fancyhead[RO,R]{1006313231}
\fancyfoot[LO,L]{}
\fancyfoot[CO,C]{\thepage}
\fancyfoot[RO,R]{}

\makeatletter
\renewcommand*\env@matrix[1][*\c@MaxMatrixCols c]{%
  \hskip -\arraycolsep
  \let\@ifnextchar\new@ifnextchar
  \array{#1}}
\makeatother

\newcommand\myalign[1]{\alignShortstack{\strut#1\strut}}
\usepackage{tabstackengine}
\TABstackMath
\TABstackMathstyle{\displaystyle}

\newcount\arrowcount
\newcommand\arrows[1]{
    \global\arrowcount#1
    \ifnum\arrowcount>0
            \begin{matrix}[c]
            \expandafter\nextarrow
    \fi
}

\newcommand\nextarrow[1]{
    \global\advance\arrowcount-1
    \ifx\relax#1\relax\else \xrightarrow{#1}\fi
    \ifnum\arrowcount=0
    \end{matrix}
    \else
    \\
    \expandafter\nextarrow
    \fi
} 
\DeclareSymbolFont{extraup}{U}{zavm}{m}{n}
\DeclareMathSymbol{\varheart}{\mathalpha}{extraup}{86}
\DeclareMathSymbol{\vardiamond}{\mathalpha}{extraup}{87}

\newcommand{\timesSmall}{{\mkern-2mu\times\mkern-2mu}}
\newcommand{\N}{\mathbb{N}}
\newcommand{\Z}{\mathbb{Z}}
\newcommand{\I}{\mathbb{I}}
\newcommand{\R}{\mathbb{R}}
\newcommand{\Q}{\mathbb{Q}}
\newcommand{\F}{\mathbb{F}}
\newcommand{\C}{\mathbb{C}}
\newcommand{\innerproduct}[2]{\left\langle #1, \ #2\right\rangle}
\newcommand{\bigInnerproduct}[2]{\big\langle #1, \ #2\big\rangle}
\newcommand{\bigbracket}[1]{\big(#1\big)}
\renewcommand{\qed}{\hfill$\blacksquare$}
\newcommand{\heart}{\ensuremath\varheartsuit}
\newcommand{\flower}{\text{\ding{95}}}

\newenvironment{proofindent}{\vspace*{1mm}\hfill\begin{minipage}{\dimexpr\textwidth-10mm}}{\end{minipage}}

%{\setstretch{1.5}$\begin{array}{r@{}>{\displaystyle}ll}& {}= & [\text{}]\\\end{array}$}
%\vspace*{1mm}\hfill\begin{minipage}{\dimexpr\textwidth-10mm}\end{minipage}

\begin{document}
\begin{center}
	\hypertarget{toc}{\LARGE \noindent \underline{\textbf{Table of contents}}}\\
\end{center}
\noindent \textbf{Problem 1:}
\vspace{1mm}
\hrule
\vspace{1mm}
\noindent\hyperlink{1.1}{(1)}\\
\hyperlink{1.2}{(2)}\\
\hyperlink{1.3}{(3)}\\
\hyperlink{1.4}{(4)}\\

\noindent \textbf{Problem 2:}
\vspace{1mm}
\hrule
\vspace{1mm}
\noindent\hyperlink{2.1}{(1)}\\
\hyperlink{2.2}{(2)}\\
\hyperlink{2.3}{(3)}\\

\noindent \textbf{Problem 3:}
\vspace{1mm}
\hrule
\vspace{1mm}
\noindent\hyperlink{3.1}{(1)}\\
\noindent\hyperlink{3.2}{(2)}\\
\noindent\hyperlink{3.3}{(3)}\\

\noindent \textbf{Problem 4:}
\vspace{1mm}
\hrule
\vspace{1mm}
\noindent\hyperlink{4.1}{(1)}\\
\hyperlink{4.2}{(2)}\\
\hyperlink{4.3}{(3)}\\[50mm]
%{\LARGE \textbf{FUCK DIAGIONAL IS WRONG}}
\newpage


% ! Problem 1 %%%%%%%%%%%%%%%%%%%%%%%%%%%%%%%%%%%%%%
{\LARGE \noindent \underline{\textbf{Problem 1.}}}\\

\hyperlink{toc}{\hypertarget{1.1}{(1)}}\\
Suppose $A \in M_{n\timesSmall n}(\mathbb{C})$ is unitarily diagonalizbale\\
\begin{tabularx}{\textwidth}{>{\centering\arraybackslash}X  >{\centering\arraybackslash}X >{\centering\arraybackslash}X}
	$ D = PAP^{-1}$ & $D \text{ is a diagonal matrix}$ & $P \text{ is a unitary matrix}$
\end{tabularx}\\\\
WTS: $AA^* = A^*A$
\\\\
%So $D = PAP^{-1} \qquad D \text{ is a diagonal matrix} \qquad P \text{ is a unitary matrix}$\\ 
%$\hspace*{2mm} A = P^{-1}DP $\\\\
{\setstretch{1.5}$\begin{array}{r@{}>{\displaystyle}ll}
		A   & {}= P^{-1}DP                                 & [\text{by given}]\vspace*{1.2mm}                  \\
		A^* & {}= \bigbracket{P^{-1}DP}^{*} \hspace*{20mm} & [\text{by definition of } A]                      \\
		    & {}= P^*D^*{P^{-1}}^{*}                       & [\text{by properties of } ^*]                     \\
		    & {}=  P^{-1}D^*P                              & [\text{by properties of } ^*]                     \\
		    & {}=  P^{-1}\overline{D^T}P                   & [\text{by definition of } ^*]                     \\
		    & {}=  P^{-1}\overline{D}P                     & [\text{since diagonal matricies are symmetrical}] \\
	\end{array}$}\\
%$\displaystyle \Longrightarrow A^* = \bigbracket{P^{-1}DP}^{*} = P^*D^*{P^{-1}}^{*} =  P^{-1}D^*P$
\\\\
{\setstretch{1.5}$\begin{array}{r@{}>{\displaystyle}ll}
		AA^* & {}= P^{-1}DP P^{-1}\overline{D}P \hspace*{20mm} & [\text{by given}]                            \\
		     & {}=  P^{-1}D\overline{D}P                       & [\text{since } PP^{-1}=I]                    \\
		     & {}=  P^{-1}\overline{D}DP                       & [\text{since diagonal matricies commute}]    \\
		     & {}=  P^{-1}\overline{D}P P^{-1}DP               & [\text{since } PP^{-1}=I]                    \\
		     & {}=  A^*A                                       & [\text{by definition of }A \text{ and } A^*] \\
	\end{array}$}\\
$\therefore A $ is normal \qed \newpage

\hyperlink{toc}{\hypertarget{1.2}{(2)}}\\
Suppose\\
\begin{tabularx}{\textwidth}{>{\centering\arraybackslash}X  >{\centering\arraybackslash}X}
	$A = UBU^*$ & $U$ is unitary or $U^* = U^{-1}$
\end{tabularx}\\\\
WTS: $A \text{ is normal } \iff \ B \text{ is normal}$\\\\
Case 1: $A$ is normal

\begin{proofindent}
	$AA^* = A^*A$
	\\\\
	WTS: $BB^* = B^*B$
	\\\\
	%$B = U^*AU$\\
	{\setstretch{1.0}$\begin{array}{r@{}>{\displaystyle}ll}
			B   & {}= U^*AU                  & [\text{by given}]\vspace*{0mm} \\
			B^* & {}= U^*A^*U \hspace*{20mm} & [\text{by definition of } B]   \\
		\end{array}$}
	\\\\
	{\setstretch{1.5}$\begin{array}{r@{}>{\displaystyle}ll}
			BB^* & {}= U^*AU U^*A^*U  \hspace*{20mm} & [\text{by definition of } B]       \\
			     & {}= U^*AA^*U                      & [\text{since } UU^* = UU^{-1} = I] \\
			     & {}= U^*A^*AU                      & [\text{since } AA^* = A^*A]        \\
			     & {}= U^*A^* UU^* AU                & [\text{since } UU^* = UU^{-1} = I] \\
			     & {}= B^*B                          & [\text{by definition of } B]       \\
		\end{array}$}\\\\
	$\therefore B$ is normal, as wanted.\\
\end{proofindent}\\\\

\noindent Case 2: $B$ is normal

\begin{proofindent}
	$BB^* = B^*B$
	\\\\
	WTS: $AA^* = A^*A$
	\\\\
	{\setstretch{1.0}$\begin{array}{r@{}>{\displaystyle}ll}
			A   & {}= UBU^*                  & [\text{by given}]\vspace*{0mm} \\
			A^* & {}= UB^*U^* \hspace*{20mm} & [\text{by definition of } B]   \\
		\end{array}$}
	\\\\
	{\setstretch{1.5}$\begin{array}{r@{}>{\displaystyle}ll}
			AA^* & {}= UBU^* UB^*U^* \hspace*{20mm} & [\text{by definition of } A]       \\
			     & {}= UBB^*U^*                     & [\text{since } UU^* = UU^{-1} = I] \\
			     & {}= UB^*BU^*                     & [\text{since } BB^* = B^*B]        \\
			     & {}= UB^*U^*UBU^*                 & [\text{since } UU^* = UU^{-1} = I] \\
			     & {}= A^*A                         & [\text{by definition of } A]       \\
		\end{array}$}\\\\
	$\therefore A$ is normal, as wanted.
\end{proofindent}\\\\
$\therefore A \text{ is normal } \iff \ B \text{ is normal}$ \newpage

\hyperlink{toc}{\hypertarget{1.3}{(3)}}\\
\begin{tabularx}{\textwidth}{>{\centering\arraybackslash}X  >{\centering\arraybackslash}X}

	Let $B = \begin{bmatrix}
			\lambda_{11} & \lambda_{12} & \lambda_{13} & \cdots & \lambda_{1n} \\[1.7mm]
			0            & \lambda_{22} & \lambda_{23} & \cdots & \lambda_{2n} \\[1.7mm]
			0            & 0            & \lambda_{33} & \cdots & \lambda_{3n} \\
			\vdots       & \vdots       & \vdots       & \ddots & \vdots       \\[1.7mm]
			0            & 0            & 0            & \cdots & \lambda_{nn} \\
		\end{bmatrix} $ & $B^* = \begin{bmatrix}
			\overline{\lambda_{11}} & 0                       & 0                       & \cdots & 0                       \\[1.7mm]
			\overline{\lambda_{12}} & \overline{\lambda_{22}} & 0                       & \cdots & 0                       \\[1.7mm]
			\overline{\lambda_{13}} & \overline{\lambda_{23}} & \overline{\lambda_{33}} & \cdots & 0                       \\
			\vdots                  & \vdots                  & \vdots                  & \ddots & \vdots                  \\[1.7mm]
			\overline{\lambda_{1n}} & \overline{\lambda_{2n}} & \overline{\lambda_{3n}} & \cdots & \overline{\lambda_{nn}}
		\end{bmatrix} $
\end{tabularx}
\\\\
Given: $BB^* = B^*B$ \\
WTS: $\lambda_{ij} = 0 \qquad $ if $i \not = j$
\\\\
{\setstretch{1.5}$\begin{array}{r@{}>{\displaystyle}ll}
	A & {}= BB^* & \text{}\\
		 & {}= B^*B & \text{} \\
		 & {}= \begin{bmatrix}
			\lambda_{11} & \lambda_{12} & \lambda_{13} & \cdots & \lambda_{1n} \\[1.7mm]
			0            & \lambda_{22} & \lambda_{23} & \cdots & \lambda_{2n} \\[1.7mm]
			0            & 0            & \lambda_{33} & \cdots & \lambda_{3n} \\
			\vdots       & \vdots       & \vdots       & \ddots & \vdots       \\[1.7mm]
			0            & 0            & 0            & \cdots & \lambda_{nn} \\
		\end{bmatrix}\begin{bmatrix}
			\overline{\lambda_{11}} & 0                       & 0                       & \cdots & 0                       \\[1.7mm]
			\overline{\lambda_{12}} & \overline{\lambda_{22}} & 0                       & \cdots & 0                       \\[1.7mm]
			\overline{\lambda_{13}} & \overline{\lambda_{23}} & \overline{\lambda_{33}} & \cdots & 0                       \\
			\vdots                  & \vdots                  & \vdots                  & \ddots & \vdots                  \\[1.7mm]
			\overline{\lambda_{1n}} & \overline{\lambda_{2n}} & \overline{\lambda_{3n}} & \cdots & \overline{\lambda_{nn}}
		\end{bmatrix}  & \text{}\\[20mm]
		&{}=\begin{bmatrix}
			\overline{\lambda_{11}} & 0                       & 0                       & \cdots & 0                       \\[1.7mm]
			\overline{\lambda_{12}} & \overline{\lambda_{22}} & 0                       & \cdots & 0                       \\[1.7mm]
			\overline{\lambda_{13}} & \overline{\lambda_{23}} & \overline{\lambda_{33}} & \cdots & 0                       \\
			\vdots                  & \vdots                  & \vdots                  & \ddots & \vdots                  \\[1.7mm]
			\overline{\lambda_{1n}} & \overline{\lambda_{2n}} & \overline{\lambda_{3n}} & \cdots & \overline{\lambda_{nn}}
		\end{bmatrix} \begin{bmatrix}
			\lambda_{11} & \lambda_{12} & \lambda_{13} & \cdots & \lambda_{1n} \\[1.7mm]
			0            & \lambda_{22} & \lambda_{23} & \cdots & \lambda_{2n} \\[1.7mm]
			0            & 0            & \lambda_{33} & \cdots & \lambda_{3n} \\
			\vdots       & \vdots       & \vdots       & \ddots & \vdots       \\[1.7mm]
			0            & 0            & 0            & \cdots & \lambda_{nn} \\
		\end{bmatrix} \hspace*{10mm}&\text{}\\
		%&{}=&[\text{}]\\
\end{array}$}
\\\\[5mm] \newpage \noindent
{\setstretch{1.5}$\begin{array}{r@{}>{\displaystyle}ll}
	A_{11} & {}= \lambda_{11} \overline{\lambda_{11}}  \hspace*{20mm} & [\text{from } BB^*]\\
	& {}= |\lambda_{11}|^2 & \text{}\\
	& {}= \sum_{i = 1}^{n} \lambda_{1i} \overline{\lambda_{1i}}   & [\text{from }B^*B]\\
	& {}= \sum_{i = 1}^{n} |\lambda_{1i}|^2   & \text{}\\

		%&{}=&[\text{}]\\
\end{array}$} \hspace*{25mm}%\\[10mm]
{\setstretch{1.5}$\begin{array}{l>{\displaystyle}r@{}>{\displaystyle}ll}
	 & \sum_{i = 1}^{n} |\lambda_{1i}|^2  & {}= |\lambda_{11}|^2 \hspace*{0mm} & \text{}                     \\
	\Longrightarrow & \sum_{i = 2}^{n}|\lambda_{1i}|^2 & {}=0                & \text{}                          \\
	\Longrightarrow & |\lambda_{1i}|^2                     & {}=0                & \text{} \\
	\Longrightarrow & |\lambda_{1i}|         & {}=0                & \text{}                          \\
	\Longrightarrow & \lambda_{1i}       & {}=0                & \text{}                            \\
\end{array}$}
\\\\[15mm]
{\setstretch{1.5}$\begin{array}{r@{}>{\displaystyle}ll}
	A_{22} & {}= \lambda_{22} \overline{\lambda_{22}} +  \lambda_{12} \overline{\lambda_{12}} \hspace*{3.5mm} & [\text{from } BB^*]\\
	& {}= |\lambda_{22}|^2 + 0 & \text{}\\
	& {}= |\lambda_{22}|^2 & \text{}\\

	& {}= \sum_{i = 2}^{n} \lambda_{2i} \overline{\lambda_{2i}}   & [\text{from }B^*B]\\
	& {}= \sum_{i = 2}^{n} |\lambda_{2i}|^2   & \text{}\\

		%&{}=&[\text{}]\\
\end{array}$} \hspace*{25mm}%\\[10mm]
{\setstretch{1.5}$\begin{array}{l>{\displaystyle}r@{}>{\displaystyle}ll}
	 & \sum_{i = 2}^{n} |\lambda_{2i}|^2  & {}= |\lambda_{22}|^2 \hspace*{0mm} & \text{}                     \\
	\Longrightarrow & \sum_{i = 3}^{n}|\lambda_{2i}|^2 & {}=0                & \text{}                          \\
	\Longrightarrow & |\lambda_{2i}|^2                     & {}=0                & \text{} \\
	\Longrightarrow & |\lambda_{2i}|         & {}=0                & \text{}                          \\
	\Longrightarrow & \lambda_{2i}       & {}=0                & \text{}                            \\
\end{array}$}
\\\\[5mm]
\noindent we can continue this pattern for all $i \in \{1, \ \cdots , \ n\}$\\
This means that for every $i \not = j,\ A_{ij} = 0$\\
$\therefore A $ is a diagonal matrix. 
\newpage
\hyperlink{toc}{\hypertarget{1.4}{(4)}}\\
Suppose $A\in M_{n\timesSmall n}(\C)$ is normal\\%  $\ \Longrightarrow \ $  $AA^* = A^*A$\\
WTS: $\exists U,D \in M_{n\timesSmall n}(\C) \text{ such that } D = UAU^{-1},\ U$ is unitary, $D$ is diagonal. 
\\\\
\\\\
By Schur's diagonalizbale lemma, $\exists B,\ U \in M_{n\timesSmall n}(\C) \text{ such that }\\ B = U^{-1}AU,\ U$ is unitary, $B$ is an upper triangular matrix.
\\\\
By \hyperlink{1.2}{(2)}, $A$ is normal $\iff B$ is normal.\\
We know $A$ is normal since it is given in the question, so this means $B$ is normal.
\\\\
by \hyperlink{1.3}{(3)}, we know that a normal upper triangular matrix is diagonal.\\
Since $B$ sastifies these conditions, this means $B$ is diagonal.
\\\\
In the end, we have $B = U^{-1}AU,\ U$ is unitary, $B$ is diagonal. \\
$\therefore $ Any normal matrix is unitarily diagonalizbale \qed
\newpage
% ! Problem 2 %%%%%%%%%%%%%%%%%%%%%%%%%%%%%%%%%%%%%%
{\LARGE \noindent \underline{\textbf{Problem 2.}}}\\

\hyperlink{toc}{\hypertarget{2.1}{(1)}}\\
Suppose $T$ is unitarily diagonalizbale with real eigenvalues. \qquad $\Longrightarrow \qquad [T]_{\mathcal{U}} $ is diagonal.\\
WTS: %$\bigInnerproduct{T(v)}{w} = \bigInnerproduct{v}{T(w)} \qquad \forall v,\ w \in V$\\
$\displaystyle [T]_{\mathcal{U}} = {[T]_{\mathcal{U}}}^*$
\\\\
Since $[T]_{\mathcal{U}}$ is diagonal, let \\[2mm]
$\hspace*{1.5mm}\displaystyle [T]_{\mathcal{U}} = \begin{bmatrix}
	\lambda_1 & 0 & \cdots & 0 \\
	0 & \lambda_2  & \cdots & 0 \\
	\vdots & \vdots & \ddots & \vdots \\
	0 & 0 & \cdots & \lambda_n \\
\end{bmatrix} \qquad \lambda_i \in \C$ 
\\\\[2mm]
$\displaystyle {[T]_{\mathcal{U}}}^* = \begin{bmatrix}
	\overline{\lambda_1} & 0 & \cdots & 0 \\
	0 & \overline{\lambda_2}  & \cdots & 0 \\
	\vdots & \vdots & \ddots & \vdots \\
	0 & 0 & \cdots & \overline{\lambda_n} \\
\end{bmatrix} = \begin{bmatrix}
	\lambda_1 & 0 & \cdots & 0 \\
	0 & \lambda_2  & \cdots & 0 \\
	\vdots & \vdots & \ddots & \vdots \\
	0 & 0 & \cdots & \lambda_n \\
\end{bmatrix}$\\[2mm]
Since the eigenvalues are real.\\
Since $T$ is hermitian, $T$ is self-adjoint. \qed
\\\\\\[30mm]

\hyperlink{toc}{\hypertarget{2.2}{(2)}}\\
Suppose $T$ is an isometry. \qquad $\Longrightarrow \qquad \innerproduct{v}{w} = \bigInnerproduct{T(v)}{T(w)}$\\
WTS: $\forall v \not = 0 \text{ such that } T(v) = \lambda v,\  \|\lambda \| = 1 $
\\\\
Let $v \in V$ be an arbitrary eigenvector.\\
{\setstretch{1.5}$\begin{array}{r@{}>{\displaystyle}ll}
	\innerproduct{v}{v} & {}= \bigInnerproduct{T(v)}{T(v)} \hspace*{20mm} & [\text{since T is an isometry}] \\
						& {}= \innerproduct{\lambda v}{\lambda v} & [\text{since $v$ us an eigenvector}]   \\
						& {}= \lambda \overline{\lambda}\innerproduct{ v}{ v} & [\text{by taking scalars out of inner product}]   \\
						& {}= |\lambda| \innerproduct{ v}{ v} & [\text{by definition of magnitude of a complex number}]   \\
\end{array}$}\\[5mm]
We know that $\innerproduct{v}{v} > 0$ by the property of positive-definite, and that $v \not = 0$ since it is an eigenvector.\\
This means in order for $\innerproduct{v}{v} = | \lambda|\innerproduct{v}{v},\ | \lambda|$ must equal 1. \qed \newpage

\hyperlink{toc}{\hypertarget{2.3}{(3)}}\\
Suppose $T$ is unitarily diagonalizbale, and all eigenvalues of $T$ has absolute value 1.\\
WTS: $T$ is an isometry.
\\\\
Let $\mathcal{U} = \{u_1,\ \cdots,\ u_n\}$ be an orthonormal basis of $V$.\\
So, for all $i \in \{1,\ \cdots,\ n\},\  u_i$ is an eigenvector.\\
Let $T(u_i) = \lambda_i u_i$
\\\\
${[T]_{\mathcal{U}}} = \begin{bmatrix}
	\lambda_1 & 0 & \cdots & 0 \\
	0 & \lambda_2  & \cdots & 0 \\
	\vdots & \vdots & \ddots & \vdots \\
	0 & 0 & \cdots & \lambda_n \\
\end{bmatrix} \qquad |\lambda_i| = 1$
\\\\
%WTS: $T$ is an isometry $\iff$ ${[T]_{\mathcal{U}}}$ is unitary $\iff$ columns of ${[T]_{\mathcal{U}}}$ are orthonormal. 
\\\\
Since for every column of ${[T]_{\mathcal{U}}}$, we have\\
i$^\text{th} \text{ column of } T = \begin{bmatrix}
	0\\\vdots\\\lambda_i\\\vdots\\0
\end{bmatrix} \leftarrow $ i$^{\text{th}}$ row
\\\\
So \big(i$^\text{th} \text{ column of } T\big)\ \cdot $ \big(j$^\text{th} \text{ column of } T\big) \\[2mm]= \left\{
	\begin{aligned}
		|\lambda_i|^2 \qquad & \text{ if } i = j &  \text{\qquad since }\lambda_i \cdot \lambda_i =  |\lambda_i|^2  \\
		0\hspace*{1.5mm} \qquad & \text{ if } i \not = j &\text{\hspace*{20mm} since all other entires are 0}\\
	\end{aligned}
	\right. \\[2mm]= \left\{
		\begin{aligned}
			1 \qquad & \hspace*{5.5mm}\text{ if } i = j\\
			0 \qquad & \hspace*{5.5mm}\text{ if } i \not = j\\
		\end{aligned}
		\right.$\\[2mm]
$\therefore$ columns of ${[T]_{\mathcal{U}}}$ are orthonormal\\
$\hspace*{4.5mm}\Longrightarrow {[T]_{\mathcal{U}}}$ is unitary\\
$\hspace*{4.5mm}\Longrightarrow T$ is an isometry. \qed
\newpage
% ! Problem 3 %%%%%%%%%%%%%%%%%%%%%%%%%%%%%%%%%%%%%%
{\LARGE \noindent \underline{\textbf{Problem 3.}}}\\

\hyperlink{toc}{\hypertarget{3.1}{(1)}}\\

\hyperlink{toc}{\hypertarget{3.2}{(2)}}\\

\hyperlink{toc}{\hypertarget{3.3}{(3)}}\\



\newpage
% ! Problem 4 %%%%%%%%%%%%%%%%%%%%%%%%%%%%%%%%%%%%%%
{\LARGE \noindent \underline{\textbf{Problem 4.}}}\\

\hyperlink{toc}{\hypertarget{4.1}{(1)}}\\

\hyperlink{toc}{\hypertarget{4.2}{(2)}}\\

\hyperlink{toc}{\hypertarget{4.3}{(3)}}\\
\end{document}