\documentclass[12pt]{article}

\usepackage[margin=1in, left=0.6in, right=0.6in]{geometry}
\usepackage{fancyhdr} % header
\usepackage{hyperref} % links
\usepackage{amsmath,amsthm,amssymb}	%math stuff
\usepackage{setspace} % increase line spacing
\usepackage[table]{xcolor} % align environment
\usepackage{changepage} % for the adjustwidth environment
\usepackage{relsize} % Scaling the font
\usepackage{algorithm} % for algorithms
\usepackage{caption} % captioning the algorithm
\usepackage[export]{adjustbox}% http://ctan.org/pkg/adjustbox
\usepackage{graphicx} \graphicspath{ {./images/} } % images
\usepackage[noend]{algpseudocode} % pseudo code
\usepackage[T1]{fontenc} % for {} in \texttt{}

\setlength{\parindent}{0pt}
\everymath{\displaystyle}

\pagestyle{fancy}
\fancyhead[LO,L]{CSCC73 A2}
\fancyhead[CO,C]{Stephen Guo, Ezzeldin Ismail}
\fancyhead[RO,R]{1006313231, 1005798861}
\fancyfoot[LO,L]{}
\fancyfoot[CO,C]{\thepage}
\fancyfoot[RO,R]{}

\begin{document}
%----------------------------------------------------------------------------------
%                              Table of Contents
%----------------------------------------------------------------------------------
\begin{center}
	\hypertarget{toc}{\LARGE \underline{\textbf{Table of Contents}}}\\
\end{center}

\hyperlink{1}{\textbf{Question 1:}}
\vspace{1mm}
\hrule
\vspace{1mm} \leavevmode \\

\hyperlink{2}{\textbf{Question 2:}}
\vspace{1mm}
\hrule
\vspace{1mm} \leavevmode \\

\newpage

%----------------------------------------------------------------------------------
%                                   Questions
%----------------------------------------------------------------------------------
\setstretch{1.2}
%----------------------------------------------------------------------------------
% !                                     1
%----------------------------------------------------------------------------------
\hyperlink{toc}{\hypertarget{1}{\LARGE \underline{\textbf{Question 1.}}}}\\

The algorithm to create a graph of the best bottleneck for $G$ is simply making a maximum spanning tree of $G$.
We will do this using Kruskal's algorithm using the disjoint-set data structure.
\begin{algorithm}
	\caption*{\textbf{Algorithm}\\Max\_Spanning\_Tree \big(\texttt{graph}: undirected graph $= (V,\ E)$\big)}\label{alg:cap}
	\begin{algorithmic}[1]
		\State let \texttt{result} = \texttt{\{\}}
		\State let \texttt{sorted-edges} = \texttt{E} sorted by bandwidth decreasing
		\State let \texttt{sorted-edges-cluster} = \texttt{[]}
		\State
		\For{\texttt{edge} in \texttt{sorted-edges}}
		  \State \texttt{sorted-edges-cluster}.append\big(MAKE-CLUSTER(\texttt{edge})\big)
		\EndFor
		\State
		\For{\texttt{edge = (u, v)} in \texttt{sorted-edges-cluster}}
			% \If{adding \texttt{edge} does not create a cycle}
		\If{\texttt{u}.cluster $\not =$ \texttt{v}.cluster}
				\State \texttt{result} = \texttt{result} $\cup$ \texttt{edge}
				\State UNION(\texttt{u}.cluster,\ \texttt{v}.cluster)
			\EndIf
		\EndFor
    \State
		\State\Return{\texttt{result}}
	\end{algorithmic}
\end{algorithm}

The algorithm \texttt{Max\_Spanning\_Tree()} is a greedy algorithm with the choice of longest edge length that does not create a cycle.\\

\texttt{Max\_Spanning\_Tree()} gives a feasible solution since we output a spanning tree with no cycles.\\

The complexity of \texttt{Max\_Spanning\_Tree()} is just $\mathcal{O}(n \log n)$ since we sort which is $\mathcal{O}(n \log n)$.
The 2 \texttt{for} loops all loop $\mathcal{O}(n)$ times and have $\mathcal{O}(1)$ operations.
Every other line is $\mathcal{O}(1)$, so the final time compelxity is $\mathcal{O}(n \log n)$\\

\newpage
\textbf{Theorem}: All spanning trees have the same number of edges\\
\textbf{Proof}:
\begin{adjustwidth}{10mm}{}
	Let $G = (V,\ E)$ be an arbitrary graph.\\

	We will show that any spanning tree of a connected graph $G$ must have $|V|-1$ edges.\\

	\textsc{Base Case}:
	\begin{adjustwidth}{10mm}{}
		For a graph of size $|V| = 1$, the a spanning tree would have no edges.
	\end{adjustwidth}
	\textsc{Induction Hypothesis}:
	\begin{adjustwidth}{10mm}{}
		Suppose for a graph $G = (V,\ E)$ with $|V| = n$, we can create a spanning tree $T$ with $n-1$ edges.
	\end{adjustwidth}
	\textsc{Induction Step}:
	\begin{adjustwidth}{10mm}{}
		\textbf{WTS}: A spanning tree $T^\prime$ for a graph $G^\prime$ with $|V^\prime| = n+1$ has $n$ edges.\\

		The spanning tree from [IH] has all nodes except for 1. Since by assumption, the graph is connected,
		then there is an edge in $G^\prime$ to the new vertex. Then use $T$ from [IH] and connect one edge in $G^\prime$ to the new node to create $T^\prime$.\\
		We now have a spanning tree $T^\prime$ for $G^\prime$ with $n$ edges. $\hspace*{\fill}\blacksquare$
	\end{adjustwidth}
\end{adjustwidth}~\\[-5mm]

\textbf{Theorem}: Greedy spanning tree $T$ given by \texttt{Max\_Spanning\_Tree()} is optimal\\
\textbf{Proof} (Greedy stays ahead):
\begin{adjustwidth}{10mm}{}
	Let $G = \{g_1,\ g_2,\ \cdots,\ g_{n-1},\ g_{n}\}$ be the spanning tree given by \texttt{Max\_Spanning\_Tree()}, ordered by add order\\
	Let $O = \{o_1,\ o_2,\ \cdots,\ o_{n-1},\ o_{n}\}$ be a maximum spanning tree that's most similar to $G$, ordered by add order\\
	We know they have to be the same size by the theorem above.\\

	\textbf{WTS}: We can change $O$ to be more similar to $G$ while keeping optimality.\\

	{
	\setstretch{1.5}
	$
		\begin{array}{r@{}>{\displaystyle}l}
			\text{Suppose\hspace*{10mm}} {g}_{1} & {}= {s}_{1}                                                         \\
			{g}_{2}                              & {}= {o}_{2}                                                         \\
			                                     & \hspace*{2.5mm} \vdots                                              \\[-3mm]
			{g}_{k-2}                            & {}= {o}_{k-2}                                                       \\
			{g}_{k-1}                            & {}= {o}_{k-1}                                                       \\
			\text{but\hspace*{10mm}}{g}_{k}      & {}\not= {o}_{k}\hspace*{9mm}\text{for some } k \in [0,\ \cdots,\ n] \\
		\end{array}
	$
	}\\

	since ${g}_{k} \not= {o}_{k}$, then $b_{o_k} \leq b_{g_k}$ because if $b_{o_k} > b_{g_k}$, then the greedy algorithm would've chosen that edge.\\

	Let $g_k = (u,\ v)$\\
	Now replace $o_k$ with $g_k$ for $O^\prime$\\

	We know that $b(P)$ with $P$ going to the verticies $u$ and $v$ in $O$ still has the best bottleneck since \texttt{Max\_Spanning\_Tree()}
	selected the biggest bandwidth edge going to them.\\

	Repeat until $O = G$\\
	$\therefore O$ becomes more similar to $G$ with every path still having the best bottleneck. $\hspace*{\fill}\blacksquare$

	% We will prove correctness by Induction.\\
	% \textsc{Base Case}:
	% \begin{adjustwidth}{10mm}{}
	% 	For a graph of size $|E| = 1$, \texttt{Max\_Spanning\_Tree()} will output that one edge which is a max spanning tree
	% \end{adjustwidth}
	% \textsc{Induction Hypothesis}:
	% \begin{adjustwidth}{10mm}{}
	% 	Suppose \texttt{Max\_Spanning\_Tree()} outputs a max spanning tree for a graph of size $|E| = k$
	% \end{adjustwidth}
	% \textsc{Induction Step}:
	% \begin{adjustwidth}{10mm}{}
	% 	\textbf{WTS}: \texttt{Max\_Spanning\_Tree()} outputs a max spaninng tree for a grpah of size $|E| = k+1$
	% \end{adjustwidth}
\end{adjustwidth}
\newpage
%----------------------------------------------------------------------------------
% !                                     2
%----------------------------------------------------------------------------------
\hyperlink{toc}{\hypertarget{2}{\LARGE \underline{\textbf{Question 2.}}}}\\\\
To prove that the degrees of a graph are valid, we just need to find one graph that supports those degrees.
The following algorithm creates a graph with the most clustering possible,
as each node tries to connect to the nodes that come after it in descending order of degrees.\\

\begin{algorithm}
	\caption*{\textbf{Algorithm}\\Validate\_Degrees \big(\texttt{graph}: undirected graph $= (V,\ E)$\big)}\label{alg:cap}
	\begin{algorithmic}[1]
		\State let \texttt{degrees} = $V$ sorted in descending order
		\If{\texttt{degrees}.[0] > $|E|$}
			\State \Return false
		\EndIf
		\State
		\For{\texttt{degree} in \texttt{degrees}}
		\State subtract 1 from the next degree elements in \texttt{degrees}
		\If{any element went below 0}
			\State \Return false
		\EndIf
		\State resort \texttt{degrees}
		\EndFor
		\State
		\State\Return{\texttt{true}}
	\end{algorithmic}
\end{algorithm}

\textbf{Complexity}: The complexity of \texttt{Validate\_Degrees()} is $\mathcal{O}(n^2\log n)$.\\

Firstly we sort the degrees which is $\mathcal{O}(n\log n)$ and then we go over every element,
and visit its right neighbours depending on its degree, and finally resorting the array.
Since we know that the max degree is bounded by $n$ due to the early return on line 2, the worst case is if its a complete graph,
in which case every element will visit every other right element, giving $\mathcal{O}(n + n\log n)$ done $n$ times which is $\mathcal{O}(n^2\log n)$.\\

\newpage
\textbf{Proof} (contradiction):
\begin{adjustwidth}{10mm}{}
	It is clear that a graph created by \texttt{Validate\_Degrees()} is valid, so if \texttt{Validate\_Degrees()} returns \texttt{true} then the input is valid.
	Let's assume that \texttt{Validate\_Degrees()} returned \texttt{false} on a valid graph.\\

	Assumption: Let $G$ be a valid graph and \texttt{Validate\_Degrees()} be false.\\

	Let \texttt{e} be the node with a degree that caused \texttt{Validate\_Degrees()} to fail (and therefore have a remaining degree greater than 0).
	That means that when attempting to connect \texttt{e} to its right neighbours to complete its degree,
	we attempted to connect it to a node with a remaining degree of 0. Therefore, there are not enough remaining nodes to for \texttt{e} to connect
	to so that it has its supposed degree.
	This is because we have already connected \texttt{e} to all of its left neighbours according to our algorithm,
	and our right neighbours are in descending order, so if we hit a neighbour with no degrees remaining,
	then none of the remaining nodes have any remaining degrees to connect to us, so the $G$ cannot be valid.
	This goes against our assumption so \texttt{Validate\_Degrees()} also returns \texttt{false} on incorrect graphs.\\

	$\therefore$ \texttt{Validate\_Degrees()} always gives the correct answer.
\end{adjustwidth}
\end{document}
