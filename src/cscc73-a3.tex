\documentclass[12pt]{article}

\usepackage[margin=1in, left=0.6in, right=0.6in]{geometry}
\usepackage{fancyhdr} % header
\usepackage{hyperref} % links
\usepackage{amsmath,amsthm,amssymb}	%math stuff
\usepackage{setspace} % increase line spacing
\usepackage[table]{xcolor} % align environment
\usepackage{changepage} % for the adjustwidth environment
\usepackage{relsize} % Scaling the font
\usepackage{algorithm} % for algorithms
\usepackage{caption} % captioning the algorithm
\usepackage[export]{adjustbox}% http://ctan.org/pkg/adjustbox
\usepackage{graphicx} \graphicspath{ {./images/} } % images
\usepackage[noend]{algpseudocode} % pseudo code
\usepackage[T1]{fontenc} % for {} in \texttt{}

% indenting in pseudocode
\algdef{SE}[SUBALG]{Indent}{EndIndent}{}{\algorithmicend\ }%
\algtext*{Indent}
\algtext*{EndIndent}

\setlength{\parindent}{0pt}
\everymath{\displaystyle}

\pagestyle{fancy}
\fancyhead[LO,L]{CSCC73 A3}
\fancyhead[CO,C]{Stephen Guo, Ezzeldin Ismail}
\fancyhead[RO,R]{1006313231, 1005798861}
\fancyfoot[LO,L]{}
\fancyfoot[CO,C]{\thepage}
\fancyfoot[RO,R]{}

\begin{document}
%----------------------------------------------------------------------------------
%                              Table of Contents
%----------------------------------------------------------------------------------
\begin{center}
	\hypertarget{toc}{\LARGE \underline{\textbf{Table of Contents}}}\\
\end{center}

\hyperlink{1}{\textbf{Question 1:}}
\vspace{1mm}
\hrule
\vspace{1mm} \leavevmode \\

\hyperlink{2}{\textbf{Question 2:}}
\vspace{1mm}
\hrule
\vspace{1mm} \leavevmode \\

\newpage

%----------------------------------------------------------------------------------
%                                   Questions
%----------------------------------------------------------------------------------
\setstretch{1.2}
%----------------------------------------------------------------------------------
% !                                     1
%----------------------------------------------------------------------------------
\hyperlink{toc}{\hypertarget{1}{\LARGE \underline{\textbf{Question 1.}}}}\\

First, we pick a random magnet. This algorithm will split the arrays until they are of size one.
Then, check every size-one arrays to see if it attracts or repels the random magnet.
We add them to the respective global linked lists, and we concatenate them in the end.
\begin{algorithm}
	\caption*{\textbf{Algorithm}\\Sort\_Mangets \big(\texttt{magnets}: \texttt{[}array of north and south magnets\texttt{]}\big)}\label{alg:cap}
	\begin{algorithmic}[1]
		\State \texttt{random\_magnet} = \texttt{magnets}\texttt{[}random()*\texttt{magnets}.length - 1\texttt{]}
		\State \texttt{attracts} = \textbf{new} linked\_list()
		\State \texttt{repels} = \textbf{new} linked\_list()
		\State
		\State \textbf{function} \texttt{split\_and\_merge} (\texttt{magnets})\Indent
			\If{\texttt{magnets}.length == 1}	
				\If{\texttt{magnets} repels \texttt{random\_magnet}}
					\State\texttt{repels}.append(\texttt{magnet})
				\Else
					\State\texttt{attracts}.append(\texttt{magnet})
				\EndIf
				\State\Return
			\EndIf
			\State
			\State \texttt{split\_and\_merge}(\texttt{magnets[}:\texttt{magnets}.length/2\texttt{]})
			\State \texttt{split\_and\_merge}(\texttt{magnets[}\texttt{magnets}.length/2:\texttt{]})
		\EndIndent
		\State
		\State \texttt{split\_and\_merge}(\texttt{magnets})
		\State \texttt{result\_magnets} = \texttt{attracts} + \texttt{[random\_magnet]} + \texttt{repels}
		\State\Return \texttt{result\_magnets}
	\end{algorithmic}
\end{algorithm}

\texttt{Sort\_Mangets()} is feasible since we return a list of magnets.

\newpage
\textbf{Proof of complexity}:\\
This algorithm consists of three parts
\begin{enumerate}
	\itemsep0em
	\item Picking a random magnet
	\item Divide and Conquer
	\item Merging the final two lists and the random magnet
\end{enumerate}
\textbf{Note}: All container elements are lists with head and tail pointers, except the initial given array of magnets.
Thus, merging and appending to the end is O(1).\\

Picking a random magnet is clearly O(1) from the magnets array.
Merging the two final linked lists with the extra element is also O(1) since everything is in a linked list.\\

The divide and conquer only works on the base case, and returns nothing. Therefore, the recurrence relation is T(n) = 2T(n/2).
Assuming T(1) is a constant, we can recurse this relation to its base case giving us T(n) = $2^{\log_2(n)}\text{T}(1) = n\text{T}(1) \in O(n)$.\\

$\therefore O(1) + O(1) + O(n) \in O(n)$\\

\textbf{Proof of Correctness} (by Induction):
\begin{adjustwidth}{10mm}{}
	\textsc{Base Case}:
	\begin{adjustwidth}{10mm}{}
		Test the single magnet against our indicator magnet:\\
		Case 1: The magnets repel so add it to the repels list\\
		Case 2: The magnets attract so add it to the attracts list\\
	\end{adjustwidth}

	Since the base case is applied over all elements then they will be in the appropriate list. So when all lists are merged all the elements will be sorted.
\end{adjustwidth}
\newpage
%----------------------------------------------------------------------------------
% !                                     2
%----------------------------------------------------------------------------------
\hyperlink{toc}{\hypertarget{2}{\LARGE \underline{\textbf{Question 2.}}}}\\

This algorithm is pretty much the counting inversions algorithm we saw in class, but we also call
it with the right array doubled.
\begin{algorithm}
	\caption*{\textbf{Algorithm}\\Sort\_and\_Count \big(\texttt{list}: \texttt{[}array of list\texttt{]}\big)}\label{alg:cap}
	\begin{algorithmic}[1]
		\State \textbf{function} \texttt{merge\_and\_count} (\texttt{left\_half}, \texttt{right\_half})\Indent
			\State \texttt{i} = 0, \texttt{j} = 0, \texttt{count} = 0
			\State \texttt{merged} = \texttt{[]}
			\State
			\While{\texttt{i} $\not =$ \texttt{left\_half}.length and \texttt{j} $\not =$ \texttt{right\_half}.length}
				\If{\texttt{j} == \texttt{right\_half}.length}
				\State \texttt{merged}.append(\texttt{left\_half[i]})
					\State \texttt{i}++
				\EndIf
				\If{\texttt{i} == \texttt{left\_half}.length}
					\State \texttt{merged}.append(\texttt{right\_half[j]})  
					\State \texttt{j}++
					\State \texttt{count}++
				\EndIf
				\If{\texttt{left\_half[i]} $\geq$ \texttt{right\_half[j]}}
					\State \texttt{merged}.append(\texttt{left\_half[i]})
				\Else
					\State \texttt{merged}.append(\texttt{right\_half[j]})  
					\State \texttt{count}++
				\EndIf
			\EndWhile
			\State\Return \texttt{count}, \texttt{merged}
		\EndIndent
		\State
		\If{\texttt{list}.length == 1}
			\State\Return \big(0, \texttt{list}\big)
		\EndIf
		\State
		\State \texttt{left\_half} = \texttt{list[:list}.length\texttt{]}
		\State \texttt{right\_half} = \texttt{list[list}.length\texttt{:]}
		\State
		\State \texttt{left\_inversions}, \texttt{left\_half} = \texttt{Sort\_and\_Count}(\texttt{left\_half})
		\State \texttt{right\_inversions}, \texttt{right\_half} = \texttt{Sort\_and\_Count}(\texttt{right\_half})
		\State
		\State \texttt{inversions}, \_ = \texttt{merge\_and\_count}\big(\texttt{left\_half}, \texttt{right\_half}.map(n => 2*n)\big)
		\State \_, \texttt{list} = \texttt{merge\_and\_count}(\texttt{left\_half}, \texttt{right\_half})
		\State\Return \texttt{left\_inversions} + \texttt{right\_inversions} + \texttt{inversions}, \texttt{list}
	\end{algorithmic}
\end{algorithm}

\newpage
\texttt{Sort\_and\_Count()} is feasible since we return a number of inversions.\\

\textbf{Proof of complexity}:\\
This algorithm is similar to \texttt{sort-and-count()} which was discussed in class.
The only difference is that we use one more \texttt{merge-and-count()} call.
This adds one more $O(n)$ call to our algorithm so overall the recurrence relation is
$$T(n) = 2T\left(\frac{n}{2}\right) + 2O(n) + O(1) = 2T\left(\frac{n}{2}\right) + O(n) + O(1) \in O(n\log n)$$
as shown in lecture.\\

\textbf{Proof of Correctness} (by Induction):
\begin{adjustwidth}{10mm}{}
	\textsc{Base Case}:
	\begin{adjustwidth}{10mm}{}
		Test with one element: return that one element has 0 inversions.
	\end{adjustwidth}
	\textsc{Induction Hypothesis}:
	\begin{adjustwidth}{10mm}{}
		Assume that our algorithm works for lists of size $0 \leq i < n$.
	\end{adjustwidth}
	\textsc{Induction Step}:
	\begin{adjustwidth}{10mm}{}
		Consider $n$:\\
		When we call \texttt{merge-and-count()} with the right array doubled, we will count inversions where $i<j,\;a_i>2a_{j}$ similar to \texttt{merge-and-count()} in lecture.
		Adding this to the counts from the left and right subarrays with element $n/2$ we will get the total amount of inversions from the entire merged
		array as smaller subsets are correct by our induction hypothesis.\\
	\end{adjustwidth}
\end{adjustwidth}
\end{document}
