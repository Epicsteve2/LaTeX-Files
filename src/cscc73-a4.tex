\documentclass[12pt]{article}

\usepackage[margin=1in, left=0.6in, right=0.6in]{geometry}
\usepackage{fancyhdr} % header
\usepackage{hyperref} % links
\usepackage{amsmath,amsthm,amssymb}	%math stuff
\usepackage{setspace} % increase line spacing
\usepackage[table]{xcolor} % align environment
\usepackage{changepage} % for the adjustwidth environment
\usepackage{relsize} % Scaling the font
\usepackage{algorithm} % for algorithms
\usepackage{caption} % captioning the algorithm
\usepackage[export]{adjustbox}% http://ctan.org/pkg/adjustbox
\usepackage{graphicx} \graphicspath{ {./images/} } % images
\usepackage[noend]{algpseudocode} % pseudo code
\usepackage[T1]{fontenc} % for {} in \texttt{}
\usepackage{mathtools} % \raisebox

\makeatletter
\newcommand{\oset}[2]{%
  {\mathop{#2}\limits^{\vbox to -.5\ex@{\kern-\tw@\ex@
   \hbox{\scriptsize #1}\vss}}}}
\makeatother

% indenting in pseudocode
\algdef{SE}[SUBALG]{Indent}{EndIndent}{}{\algorithmicend\ }%
\algtext*{Indent}
\algtext*{EndIndent}

\setlength{\parindent}{0pt}
\everymath{\displaystyle}

\pagestyle{fancy}
\fancyhead[LO,L]{CSCC73 A4}
\fancyhead[CO,C]{Stephen Guo, Ezzeldin Ismail}
\fancyhead[RO,R]{1006313231, 1005798861}
\fancyfoot[LO,L]{}
\fancyfoot[CO,C]{\thepage}
\fancyfoot[RO,R]{}

\begin{document}
%----------------------------------------------------------------------------------
%                              Table of Contents
%----------------------------------------------------------------------------------
\begin{center}
	\hypertarget{toc}{\LARGE \underline{\textbf{Table of Contents}}}\\
\end{center}

\hyperlink{1}{\textbf{Question 1:}}
\vspace{1mm}
\hrule
\vspace{1mm} \leavevmode \\

\hyperlink{2}{\textbf{Question 2:}}
\vspace{1mm}
\hrule
\vspace{1mm} \leavevmode \\

\newpage

%----------------------------------------------------------------------------------
%                                   Questions
%----------------------------------------------------------------------------------
\setstretch{1.2}
%----------------------------------------------------------------------------------
% !                                     1
%----------------------------------------------------------------------------------
\hyperlink{toc}{\hypertarget{1}{\LARGE \underline{\textbf{Question 1.}}}}\\

% The subproblem we will be looking at the the optimal cost for week $i$.\\
The subproblems we will be looking at are the best laundry services to get for loads up to week $i-1$ or $i-3$
where the best services are decided by minimizing cost using $\min\big(\texttt{price\_table}[i-1]\big)$.\\
For a set of loads $l_0, \cdots , l_i$, we can find can either choose $L_1$ or $L_2$ for $l_i$. If we:
\begin{itemize}
	\itemsep0em
	\item Choose $L_1$ then we add $r\times l_i$ and the optimal cost for $l_0, \cdots , l_{i-1}$
	\item Choose $L_2$ then we add $3\times w$ and the optimal cost for $l_0, \cdots , l_{i-3}$
\end{itemize}


The recurrence relation for the optimal cost for $l_i$ is
$$
	\text{Opt}(i) = \left\{\begin{aligned}
		        & 0                             &                                  & \hspace*{2em}\text{if } i = 0               \\
		        & l_i\times r + \text{Opt}(i-1) &                                  & \hspace*{2em}\text{if } i = 1 \text{ or } 2 \\
		        & \text{min}
		\left(
		\begin{aligned}
				 & l_i\times r + \text{Opt}(i-1),\ \\
				 & 3w + \text{Opt}(i-3)
			\end{aligned}
		\right) &                               & \hspace*{2em}\text{if } i \geq 3                                               \\
	\end{aligned}\right.
$$
To find the optimal schedule, we can trace backwards. If we chose $L_1$, go back one load. If we chose $L_2$, go back 3 loads.
\begin{algorithm}
	\caption*{\textbf{Algorithm}\\Choose\_Laundromats \big(\texttt{loads}: \texttt{[}array of load size per week\texttt{], \texttt{r}:$\ L_1$ rate, \texttt{w}:$\ L_2$ rate}\big)}\label{alg:cap}
	\begin{algorithmic}[1]
		% \Statex\hspace*{-1.0em}\textbf{Input} test
		% \Statex\hspace*{-1.0em}\textbf{Output} test
		% \Statex
		\State \texttt{price\_table} = \texttt{[0*n]}
		\State \texttt{schedule} = \texttt{[]}
		\State
		\For{\texttt{i} = 1 to n}
			\If{\texttt{i} == 1 or \texttt{i} == 2}
				\State \texttt{price\_table[i]} = min($l_i r$ + \texttt{price\_table[i-1]}, $3 w$)
			\Else
				\State \texttt{price\_table[i]} = min($l_i r$ + \texttt{price\_table[i-1]}, $3 w$ + \texttt{price\_table[i-3]})
			\EndIf
		\EndFor
		\State
		\State \texttt{bt\_index} = n
		\While{\texttt{bt\_index} $\geq$ 1}
			\If{\texttt{price\_table[\texttt{bt\_index}-1]} + $l_\text{\texttt{bt\_index-1}}r$ = \texttt{price\_table[bt\_index]}}
				\State\texttt{schedule}.prepend($L_1$)
				\State\texttt{bt\_index} $\mathrel{-}=$ 1
			\Else
				\State\texttt{schedule}.prepend($L_2$)
				\State\texttt{bt\_index} $\mathrel{-}=$ 3
			\EndIf
		\EndWhile
		\State
		\State\Return \texttt{schedule}
	\end{algorithmic}
\end{algorithm}

\newpage
\textbf{Proof of Correctness} (by Induction):
\begin{adjustwidth}{10mm}{}
	\textsc{Base Case}:
	\begin{adjustwidth}{10mm}{}
		If $i = 0$, return 0, since we can't choose any laundromats\\
		If $i = 1$, return $rl_1$\, since we can't choose $L_2$\\
		If $i = 2$, return $rl_1 + rl_2$, since we can't choose $L_2$
	\end{adjustwidth}
	\textsc{Induction Hypothesis}:
	\begin{adjustwidth}{10mm}{}
		Assume Opt$(j)$ returns the optimal load costs for $0\leq j < i$
	\end{adjustwidth}
	\textsc{Induction Step}:
	\begin{adjustwidth}{10mm}{}
		Consider $l_i$:\\
		If $l_i$ should use $L_1$, then the new cost will be $rl_i + \text{Opt}(i-1)$\\
		If $l_i$ should use $L_2$, then the new cost will be $3w + \text{Opt}(i-3)$\\
		Since we pick the minimum of those two values, and they are the only two options, the recurrence relation uses the minimum cost of the values by the induction hypothesis.
	\end{adjustwidth}
\end{adjustwidth}~\\[-8mm]

For the time complexity we have $\mathcal{O}(n)$. This is because we have two for loops at lines 4 and 11 which loop from 1 to $n$ and from $n$ to 1.
In each of these loops we only do $\mathcal{O}(1)$ operations such as $\min$, addition, multiplication, and array look up and array assignment,
so in total we use $n\mathcal{O}(1) \in \mathcal{O}(n)$ time.\\

For space complexity we have $\mathcal{O}(n)$. This is because we have the input array which is $\mathcal{O}(n)$ and we initialize an array with $n$ zeros,
and we create an empty array for our return result, which can at most have $n$ elements, as each load cannot use more than one laundry service.
All operations are done on these arrays or using integer variables, so in total we use $3n\in \mathcal{O}(n)$ space.
\newpage
%----------------------------------------------------------------------------------
% !                                     2
%----------------------------------------------------------------------------------
\hyperlink{toc}{\hypertarget{2}{\LARGE \underline{\textbf{Question 2.}}}}\\

The subproblem we will be looking at is the longest increasing trend that ends at index $i$.\\
We can simply take the max value of longest increasing trend that ends at index $i$ and return it.\\

To compute the solution using the subproblems, we have to is to look at all previous indexes.
We want the current price $S[i]$ to be bigger than the longest increasing trend's price $S[j]$ for some $j$ before $i$.
This means that we can add $S[i]$ to the end of the longest increasing trend, so we have the length of that +1\\

Suppose the sequence $S$ starts at index 1. The recurrence relation is
$$
	\text{Opt}(i) =
	\left\{\begin{aligned}
		 & \hspace*{4.0mm} 0                                                                                                             &  & \hspace*{2em}\text{if } i = 0 \\
		 & \max\left(\raisebox{4mm}{$\underset{S[i] > S[j]}{\underset{1\leq j< i}{\text{max}}}$}\ \big(\text{Opt}[j]\big) + 1,\ 1\right) &  & \hspace*{2em}\text{otherwise} \\
	\end{aligned}\right.
$$

\begin{algorithm}
	\caption*{\textbf{Algorithm}\\Find\_Longest\_Increasing\_Trend \big(\texttt{prices}: \texttt{[}array of price quotes\texttt{]}\big)}\label{alg:cap}
	\begin{algorithmic}[1]
	\State \texttt{lit\_at\_index} = \texttt{[}1 * \texttt{prices}.length\texttt{]}
	% \State \texttt{max} = 1;
	\State
	\For{\texttt{i} in \texttt{prices}}
		\For{j in \texttt{prices[:i]}}
		\If{\texttt{prices[j]} < \texttt{prices[i]} and \texttt{lit\_at\_index[i]} $\leq$ \texttt{lit\_at\_index[j]}}
		\State\texttt{lit\_at\_index[i]} = \texttt{lit\_at\_index[j]} + 1
		\EndIf
		\EndFor
	\EndFor
	\State
	\State\Return max(\texttt{lit\_at\_index})
	\end{algorithmic}
\end{algorithm}

The time complexity of \texttt{Find\_Longest\_Increasing\_Trend()} is $\mathcal{O}(n^2)$, and the space complexity is $\mathcal{O}(n)$.\\
We have a for loop that iterates over \texttt{prices} which is of size $n$, and the inner for loop runs from 1 to $n$.
Every other line is $\mathcal{O}(1)$, so we have a final time complexity of $\mathcal{O}(n^2)$.\\

For space complexity, we just keep a memoization array of size $n$, so the final space complexity is $\mathcal{O}(n)$.\\

\newpage
\textbf{Proof of Correctness} (by Induction):
\begin{adjustwidth}{10mm}{}
	\textsc{Base Case}:
	\begin{adjustwidth}{10mm}{}
		If $i = 0$, return 0, since there are no numbers for the increasing subsequence.
	\end{adjustwidth}
	\textsc{Induction Hypothesis}:
	\begin{adjustwidth}{10mm}{}
		Assume Opt$(j)$ returns the optimal solution for $0 \leq j < i$
	\end{adjustwidth}
	\textsc{Induction Step}:
	\begin{adjustwidth}{10mm}{}
		Consider Opt$(i)$\\
		We go over all the elements before $i$ and select the largest possible sequence with an element smaller than use +1.
		Since all the largest possible sequence for each element is optimal by our induction hypothesis,
		retrieving the largest one +1 will give us the maximal strictly increasing sequence.
	\end{adjustwidth}
\end{adjustwidth}
\end{document}
