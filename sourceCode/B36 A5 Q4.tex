\documentclass[12pt]{article}
	
\usepackage[margin=1in]{geometry}	
\usepackage{amsmath,amsthm,amssymb,scrextend}			
\usepackage{fancyhdr}				
\usepackage{graphicx}				
\usepackage{cancel}					
\usepackage{changepage}
\usepackage{color,soul}
\usepackage{enumitem}
\usepackage{array}
\usepackage{hyperref}
\usepackage{setspace}
\usepackage{tabularx}
\usepackage{pifont}
\usepackage[table,x11names]{xcolor}
\usepackage{colortbl}

\pagestyle{fancy}
\fancyhead[LO,L]{CSCB36 Assignment 5}
\fancyhead[CO,C]{Stephen Guo}
\fancyhead[RO,R]{1006313231}
\fancyfoot[LO,L]{}
\fancyfoot[CO,C]{\thepage}
\fancyfoot[RO,R]{}

\makeatletter
\renewcommand*\env@matrix[1][*\c@MaxMatrixCols c]{%
  \hskip -\arraycolsep
  \let\@ifnextchar\new@ifnextchar
  \array{#1}}
\makeatother

\newcommand\myalign[1]{\alignShortstack{\strut#1\strut}}
\usepackage{tabstackengine}
\TABstackMath
\TABstackMathstyle{\displaystyle}

\newcount\arrowcount
\newcommand\arrows[1]{
    \global\arrowcount#1
    \ifnum\arrowcount>0
            \begin{matrix}[c]
            \expandafter\nextarrow
    \fi
}

\newcommand\nextarrow[1]{
    \global\advance\arrowcount-1
    \ifx\relax#1\relax\else \xrightarrow{#1}\fi
    \ifnum\arrowcount=0
    \end{matrix}
    \else
    \\
    \expandafter\nextarrow
    \fi
} 
\DeclareSymbolFont{extraup}{U}{zavm}{m}{n}
\DeclareMathSymbol{\varheart}{\mathalpha}{extraup}{86}
\DeclareMathSymbol{\vardiamond}{\mathalpha}{extraup}{87}

\newcommand{\timesSmall}{{\mkern-2mu\times\mkern-2mu}}
\newcommand{\N}{\mathbb{N}}
\newcommand{\Z}{\mathbb{Z}}
\newcommand{\I}{\mathbb{I}}
\newcommand{\R}{\mathbb{R}}
\newcommand{\Q}{\mathbb{Q}}
\newcommand{\F}{\mathbb{F}}
\newcommand{\C}{\mathbb{C}}
\newcommand{\innerproduct}[2]{\left\langle #1, \ #2\right\rangle}
\newcommand{\bigbracket}[1]{\big(#1\big)}
\renewcommand{\qed}{\hfill$\blacksquare$}
\newcommand{\heart}{\ensuremath\varheartsuit}
\newcommand{\flower}{\text{\ding{95}}}

%%%%%%%%%%%%%%%%%%%%%%%%%% Define some useful colors %%%%%%%%%%%%%%%%%%%%%%%%%%
\definecolor{ocre}{RGB}{243,102,25}
\definecolor{epicPurple}{RGB}{162, 113, 255}
\definecolor{epicOrange}{RGB}{254, 113, 69}
\definecolor{mygray}{RGB}{243,243,244}
\definecolor{deepGreen}{RGB}{26,111,0}
\definecolor{shallowGreen}{RGB}{235,255,255}
\definecolor{deepBlue}{RGB}{61,124,222}
\definecolor{shallowBlue}{RGB}{235,249,255}
%%%%%%%%%%%%%%%%%%%%%%%%%%%%%%%%%%%%%%%%%%%%%%%%%%%%%%%%%%%%%%%%%%%%%%%%%%%%%%%

\newenvironment{proofindent}{\vspace*{1mm}\hfill\begin{minipage}{\dimexpr\textwidth-10mm}}{\end{minipage}}

\begin{document}
{\LARGE \noindent \underline{\textbf{Problem 4.}}}\\

(a) 
\begin{center}
    \begin{tabular}{|c|c|c|| ccc>{\columncolor[gray]{0.8}}cccc|}
        \hline
        $x$ & $y$ & $z$ &$\big(\neg x$&$\to$&$ (y \wedge z) \big)$&$ \wedge $&$\big(\neg y$&$ \to $&$ (x \wedge z)\big)$\\ 
        \hline \hline 
        0&0&0&  1&0&0&0&1&0&0 \\   
        0&0&1&  1&0&0&0&1&0&0 \\
        0&1&0&  1&0&0&0&0&1&0 \\  
        0&1&1&  1&1&1&1&0&1&0 \\ 
        \hline 
        1&0&0&  0&1&0&0&1&0&0 \\  
        1&0&1&  0&1&0&1&1&1&1 \\  
        1&1&0&  0&1&0&1&0&1&0 \\  
        1&1&1&  0&1&1&1&0&1&1 \\  
        \hline 
    \end{tabular}
\end{center}

\newpage
(b)\\
To get the DNF form, we highlight all rows that evaluate to true.
\begin{center} \begin{tabular}{|c|c|c|| ccccccc|}
    \hline
    $x$ & $y$ & $z$ &$\big(\neg x$&$\to$&$ (y \wedge z) \big)$&$ \wedge $&$\big(\neg y$&$ \to $&$ (x \wedge z)\big)$\\ 
    \hline \hline 
    0&0&0&  1&0&0&0&1&0&0 \\   
    0&0&1&  1&0&0&0&1&0&0 \\
    0&1&0&  1&0&0&0&0&1&0 \\  
    \rowcolor{epicPurple}0&1&1&  1&1&1&1&0&1&0 \\ 
    \hline 
    1&0&0&  0&1&0&0&1&0&0 \\  
    \rowcolor{epicPurple}1&0&1&  0&1&0&1&1&1&1 \\  
    \rowcolor{epicPurple}1&1&0&  0&1&0&1&0&1&0 \\  
    \rowcolor{epicPurple}1&1&1&  0&1&1&1&0&1&1 \\  
    \hline 
\end{tabular} \end{center}
So we end up with:
\[(\neg x \wedge y \wedge z)\vee (x \wedge \neg y \wedge z)\vee  (x \wedge y \wedge \neg z)\vee  (x \wedge y \wedge z)\\
\]

\newpage
(c)\\
To get the CNF form, we negate our original expression.
\begin{center}
    \begin{tabular}{|c|c|c|| c|}
        \hline
        $x$ & $y$ & $z$ &$\neg \Big(\bigbracket{\neg x \to (y \wedge z)} \wedge \bigbracket{\neg y \to (x \wedge z)}\Big)$\\ 
        \hline \hline 
        \rowcolor{epicOrange}0&0&0&  1 \\   
        \rowcolor{epicOrange}0&0&1&  1 \\
        \rowcolor{epicOrange}0&1&0&  1 \\  
        0&1&1&  0 \\ 
        \hline 
        \rowcolor{epicOrange}1&0&0&  1 \\  
        1&0&1&  0 \\  
        1&1&0&  0 \\  
        1&1&1&  0 \\  
        \hline 
    \end{tabular}
\end{center}
Now negate the DNF for this truth table.
\\\\
$\neg\bigbracket{(\neg x \wedge \neg y \wedge \neg z) \vee (\neg x \wedge \neg y \wedge z) \vee (\neg x \wedge y \wedge \neg z) \vee (x \wedge \neg y \wedge \neg z)}$\\
{\setstretch{1.5}$\begin{array}{rc>{\displaystyle}ll}
    &\text{\textsc{leqv}}&
    \neg(\neg x \wedge \neg y \wedge \neg z) \wedge \neg(\neg x \wedge \neg y \wedge z) \wedge \neg(\neg x \wedge y \wedge \neg z) \wedge \neg(x \wedge \neg y \wedge \neg z)&\text{}\\
    &\text{\textsc{leqv}}&
    (\neg \neg x \vee \neg\neg y \vee \neg \neg z) \wedge (\neg\neg  x \vee \neg \neg y \vee \neg z) \wedge (\neg \neg x \vee \neg y \vee \neg \neg z) \wedge (\neg  \vee \neg \neg y \vee \neg \neg z)&\text{}\\
    &\text{\textsc{leqv}}&
    (x \vee  y \vee  z) \wedge (  x \vee  y \vee \neg z) \wedge ( x \vee \neg y \vee  z) \wedge (\neg  \vee  y \vee  z)&\text{}\\
    \end{array}$}
\\[50mm]
\textbf{Eaiser way:} (I watched this part of the lecture after I did this already D:)\\
Highlight all rows that evaluate to false. Then negate all of the variables. 
\begin{center} \begin{tabular}{|c|c|c|| ccccccc|}
    \hline
    $x$ & $y$ & $z$ &$\big(\neg x$&$\to$&$ (y \wedge z) \big)$&$ \wedge $&$\big(\neg y$&$ \to $&$ (x \wedge z)\big)$\\ 
    \hline \hline 
    \rowcolor{epicOrange}0&0&0&  1&0&0&0&1&0&0 \\   
    \rowcolor{epicOrange}0&0&1&  1&0&0&0&1&0&0 \\
    \rowcolor{epicOrange}0&1&0&  1&0&0&0&0&1&0 \\  
    0&1&1&  1&1&1&1&0&1&0 \\ 
    \hline 
    \rowcolor{epicOrange}1&0&0&  0&1&0&0&1&0&0 \\  
    1&0&1&  0&1&0&1&1&1&1 \\  
    1&1&0&  0&1&0&1&0&1&0 \\  
    1&1&1&  0&1&1&1&0&1&1 \\  
    \hline 
\end{tabular} \end{center}
So we end up with:
\[(x \vee  y \vee  z) \wedge (  x \vee  y \vee \neg z) \wedge ( x \vee \neg y \vee  z) \wedge (\neg  \vee  y \vee  z)\\
\]
\newpage (d)\\
$\bigbracket{\neg x \to (y \wedge z)} \wedge \bigbracket{\neg y \to (x \wedge z)} $\\
{\setstretch{1.5}$\begin{array}{rc>{\displaystyle}ll}
        \hspace*{10mm}& \text{\textsc{leqv}} & \bigbracket{\neg\neg x \vee (y \wedge z)} \wedge \bigbracket{\neg \neg  y \vee (x \wedge z)}   & [\to \text{ law}] \\
        & \text{\textsc{leqv}} & \bigbracket{ x \vee (y \wedge z)} \wedge \bigbracket{ y \vee (x \wedge z)}  & [\text{double negation}] \\
        & \text{\textsc{leqv}} & \bigbracket{ (x \vee y) \wedge (x \vee z)} \wedge \bigbracket{ (y \vee x) \wedge (y \vee z)}  \hspace*{20mm} & [\text{distributive laws}] \\
        & \text{\textsc{leqv}} &  (x \vee y) \wedge (x \vee z) \wedge  (y \vee x) \wedge (y \vee z)   & [\text{associative laws}] \\
        & \text{\textsc{leqv}} &  \big((x \vee y) \wedge  (y \vee x)\big) \wedge (x \vee z)  \wedge (y \vee z)   & [\text{by reordering}] \\
        & \text{\textsc{leqv}} &  (x \vee y)  \wedge (x \vee z)  \wedge (y \vee z)   & [\text{idempotency laws}] \\
\end{array}$}
\end{document}