\documentclass[12pt]{article}
	
\usepackage[margin=1in]{geometry}	
\usepackage{amsmath,amsthm,amssymb,scrextend}			
\usepackage{fancyhdr}				
\usepackage{graphicx}				
\usepackage{cancel}					
\usepackage{changepage}
\usepackage{color,soul}
\usepackage{enumitem}
\usepackage{array}

\pagestyle{fancy}
\fancyhead[LO,L]{CSCB36 Assignment 2}
\fancyhead[CO,C]{Stephen Guo}
\fancyhead[RO,R]{1006323132}
\fancyfoot[LO,L]{}
\fancyfoot[CO,C]{\thepage}
\fancyfoot[RO,R]{}

\renewcommand{\qed}{\hfill$\blacksquare$}

\begin{document}
% ! Problem 5a
{\LARGE \noindent \underline{\textbf{Problem 5(a)}}}\\

\noindent \textbf{Proof of correctness for} \textsc{MaxOdd($L$)}:
\\\\
For a natural $n \geq 0$, we define predicate $Q(n)$ as follows:
\begin{itemize}[leftmargin=20mm]
	\item [$Q(n):$]If $L$ is a list of integers, $n = \text{len}(L)$, then \textsc{MaxOdd($L$)}
	      terminates and returns the largest integer in $O(L)$ if $O(L)$ is non-empty. Otherwise, return 0.
\end{itemize}
By complete induction, we will prove $Q(n)$ holds for all $n \geq 0$ as follows:\\
\vspace*{1mm}

\noindent \textsc{Basis:} % ! Base cases
\vspace*{1mm}

\hfill\begin{minipage}{\dimexpr\textwidth-10mm}
	Let n = 0
	\vspace*{1mm}

	\hfill\begin{minipage}{\dimexpr\textwidth-10mm}
		Then $L = \varnothing$ which means $O(L) = \varnothing$\\
		$\Longrightarrow$ \textsc{MaxOdd}$(L) = 0$ by Postcondition.
		\\\\
		Line 1 is run since len($L$) $\leq 1$, and line 2 is run which terminates and\\ returns 0 as wanted.\\
	\end{minipage}\\\\
	Let n = 1
	\vspace*{1mm}

	\hfill\begin{minipage}{\dimexpr\textwidth-10mm}
		then $L$ is a 1-element list which has no odd indicies.
		By the definition of $O(L)$, it is empty.\\
		$\Longrightarrow$ \textsc{MaxOdd}$(L) = 0$ by Postcondition.
		\\\\
		Line 1 is run since len($L$) $\leq 1$, and line 2 is run which terminates and returns 0 as wanted.
	\end{minipage}
\end{minipage}
\\\\\\
\noindent \textsc{Induction Step:} let $n > 1$ % ! Induction step
\vspace*{1mm}

\hfill\begin{minipage}{\dimexpr\textwidth-10mm}
	Let max$(L)$ return 0 if a list of integers $L$ is empty.\\
	Suppose $Q(j)$ holds whenever $0 \leq j < n$. \qquad \qquad \qquad \,\,\,\,\textbf{[IH]}\\
	WTP: $Q(n)$ holds.
	\\\\
	Since $\text{len}(L) = n > 1$, lines 4-8 is run.\\
	There are 3 cases to consider:\\
	Case 1: $L[1]$ is even
	\vspace*{1mm}

	\hfill\begin{minipage}{\dimexpr\textwidth-10mm} % ! Case 1
		$O(L[:2]) = \varnothing$  by definition of $O(L)$\\
		$result = m$ by line 5.\\
		By line 8, $result$ is returned and program terminates\\
		$\Longrightarrow$  \textsc{MaxOdd}$(L) = m$
		\\\\
		\emph{WTS:} \textsc{MaxOdd}$(L) = \text{max}(O(L))$
		\\\\
		$\begin{array}{l@{}>{\displaystyle}ll}
				\text{\textsc{MaxOdd}}(L) & {}=  m                                                    &                                       \\
				                          & {}=  \text{\textsc{MaxOdd}}(L[2:])                        & \text{by line 4}                      \\
				                          & {}=  \text{max}(O(L[2:]))                                 & \text{by \textbf{[IH]}}               \\% 0 \leq len(L[2:]) < n \\
				                          & {}=  \text{max}(\varnothing \ \cup \ O(L[2:]))            & \text{by union properties}            \\
				                          & {}=  \text{max}(O(L[:2]) \ \cup \ O(L[2:])) \qquad \qquad & \text{since $O(L[:2]) = \varnothing$} \\
				                          & {}=  \text{max}(O(L))                                     & \text{by union properties}            \\
			\end{array}$
		$\Longrightarrow Q(n)$ holds as wanted.\\
	\end{minipage}
	\\\\
	Case 2: $L[1]$ is odd, $m = 0$
	\vspace*{1mm}

	\hfill\begin{minipage}{\dimexpr\textwidth-10mm} % ! Case 2
		$O(L[2:]) = \varnothing$ by Postcondition of \textsc{MaxOdd}$(L[2:])$ from line 4 which terminates by \textbf{[IH]}.\\
		$O(L[:2])$ is a 1-element list containing $L[1]$.\\
		$result = L[1]$ by line 6.\\
		By line 8, $result$ is returned and program terminates\\
		$\Longrightarrow$  \textsc{MaxOdd}$(L) = L[1]$
		\\\\
		\emph{WTS:} \textsc{MaxOdd}$(L) = \text{max}(O(L))$
		\\\\
		$\begin{array}{l@{}>{\displaystyle}ll}
				\text{\textsc{MaxOdd}}(L) & {}= L[1]                                                 &                                   \\
				                          & {}= \text{max}(L[1])                                     & \text{by definition of max}(L)    \\
				                          & {}= \text{max}(O(L[:2]))                                 & \text{since $O(L[:2]) = L[1]$}    \\
				                          & {}= \text{max}(O(L[:2]) \ \cup \ \varnothing)            & \text{by union properties}        \\
				                          & {}= \text{max}(O(L[:2]) \ \cup \ O(L[2:])) \qquad \qquad & \text{since $O(L[2:]) = \varnothing$} \\
				                          & {}= \text{max}(O(L))                                     & \text{by union properties}        \\
			\end{array}$
		$\Longrightarrow Q(n)$ holds as wanted.\\
	\end{minipage}
\end{minipage}
\newpage
\hfill\begin{minipage}{\dimexpr\textwidth-10mm}
	%\\\\
	Case 3: $L[1]$ is odd, $m \not = 0$
	\vspace*{1mm}

	\hfill\begin{minipage}{\dimexpr\textwidth-10mm} % ! Case 3
		$O(L[2:])$ is non-empty.\\
		$O(L[:2])$ is a 1-element list containing $L[1]$\\
		$result = \text{max}(L[1], m)$ by line 7\\
		By line 8, $result$ is returned and program terminates\\
		$\Longrightarrow$  \textsc{MaxOdd}$(L) = \text{max}(L[1], m)$
		\\\\
		\emph{WTS:} \textsc{MaxOdd}$(L)=\text{max}(O(L))$
		\\\\
		$\begin{array}{l@{}>{\displaystyle}ll}
				\text{\textsc{MaxOdd}}(L) & {}= \text{max}(L[1], m)                                  &                                \\
				                          & {}= \text{max}(L[1], \text{\textsc{MaxOdd}}(L[2:]))      & \text{by line 4}               \\
				                          & {}= \text{max}(L[1], \text{max}(O(L[2:])))               & \text{by \textbf{[IH]}}        \\
				                          & {}= \text{max}(O(L[:2]), \text{max}(O(L[2:])))           & \text{since $O(L[:2]) = L[1]$} \\
				                          & {}= \text{max}(O(L[:2]) \ \cup \ O(L[2:])) \qquad \qquad & \text{by definition of max}(L) \\
				                          & {}= \text{max}(O(L))                                     & \text{by union properties}     \\
			\end{array}$
		$\Longrightarrow Q(n)$ holds as wanted.
	\end{minipage}
\end{minipage}
\\\\
$\therefore$ by Complete Induction, $Q(n)$ holds for all $n \in \mathbb{N}$ \qed
\end{document}